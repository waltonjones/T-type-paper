\subsection*{Cloning DmCa\textsubscript{v}3}

We generated a full-length DmCa\textsubscript{v}3 (CG15899) cDNA by piecemeal PCR amplification.
Total RNA extracted from adult heads using Trizol reagents (Invitrogen) was reverse transcribed using RevertAid First Strand cDNA Synthesis Kit (Fermentas).
Six adjacent DNA fragments that cover the entire DmCa\textsubscript{v}3 cDNA were obtained by PCR amplification. 
Primer sets were designed based on the FlyBase annotation. \textcolor{red}{Which version of FlyBase? FlyBase changes over time.}
Hind III and Xba I sites were inserted at the 5' end of clone 1 and 3' end of clone 6, respectively.
Primer sets: clone 1 (5'-CGAGATAAGCTTAAAATGCTGCCACAGCCA-3', 5'-GCATCAGACTACATCGCTGTC-3'), clone 2 (5'-CTGGACACGCTGCCCATGCTG-3', 5'-TTCCAGCTCCTCCACTTGCAC-3'), clone 3 (5'-CAACGGTGGCTCCAACAGTCG-3', 5'-CCACTGGCGGAAGCTCATGCC-3'), clone 4 (5'-GCCACGCCTCTCCAAGATCCG-3', 5'-GACGATAAGAGCGTTTGCACG-3'), clone 5 (5'-TCTGAAACTAGTCGTGCAAAC-3', 5'-TGGAAGTACTGGACGGTCTGC-3'), and clone 6 (5'-AATCCCAGCCTGACCAGCTCG-3', 5'-TCTAGATTAGTCCATGGAGGATTGGGGTGA-3').
Amplified PCR fragments were sequenced and assembled into pBlueScript II KS (+) using sequential restriction enzyme digests.
Clones 2 and 3 contained isoform-specific exons.
Of the combinations that were amplified by PCR, we chose to proceed to assembling the RB and RC isoforms.
Frequent A to G RNA editing sites were identified in clones 3 and 5.
One RNA editing site in clone 3 (5'-AGTTCAGAGC-3') was reversed to A by site-directed mutagenesis.
Since there were too many edited sites in clone 5, we reverted to using genomic DNA a as template for clone 5 instead of cDNA.
\textcolor{red}{Why did you do this? If it came from cDNA and you found the same edits several times, then what makes you think it wasn't supposed to be there? Maybe this was why your UAS rescue didn't work?}
The final assembled full-length cDNAs were cut with HindIII / XbaI and subcloned into pcDNA3-HE3 downstream of the 5'-UTR from the \emph{Xenopus laevis} $\beta$-globin gene to improve expression in \emph{Xenopus} oocytes.
