\subsection*{Cloning DmCa\textsubscript{v}3}

We generated a full-length DmCa\textsubscript{v}3 (CG15899) cDNA by piecemeal PCR amplification.
Total RNA extracted from adult heads using Trizol reagents (Invitrogen) was reverse transcribed using RevertAid First Strand cDNA Synthesis Kit (Fermentas).
Six adjacent DNA fragments that cover the entire DmCa\textsubscript{v}3 cDNA were obtained by PCR amplification. 
Primer sets were designed based on the FlyBase (FB2011\_07) annotation for DmCa\textsubscript{v}3. 
Hind III and Xba I sites were inserted at the 5' end of fragment 1 and 3' end of fragment 6, respectively.
Primer sets: fragment 1 (\seqsplit{5'-CGAGATAAGCTTAAAATGCTGCCACAGCCA-3'}, \seqsplit{5'-GCATCAGACTACATCGCTGTC-3'}), fragment 2 (\seqsplit{5'-CTGGACACGCTGCCCATGCTG-3'}, \seqsplit{5'-TTCCAGCTCCTCCACTTGCAC-3'}), fragment 3 (\seqsplit{5'-CAACGGTGGCTCCAACAGTCG-3'}, \seqsplit{5'-CCACTGGCGGAAGCTCATGCC-3'}), fragment 4 (\seqsplit{5'-GCCACGCCTCTCCAAGATCCG-3'}, \seqsplit{5'-GACGATAAGAGCGTTTGCACG-3'}), fragment 5 (\seqsplit{5'-TCTGAAACTAGTCGTGCAAAC-3'}, \seqsplit{5'-TGGAAGTACTGGACGGTCTGC-3'}), and fragment 6 (\seqsplit{5'-AATCCCAGCCTGACCAGCTCG-3'}, \seqsplit{5'-TCTAGATTAGTCCATGGAGGATTGGGGTGA-3'}).
Amplified PCR fragments were sequenced and assembled into pBlueScript II KS (+) using sequential restriction enzyme digests.
Clones 2 and 3 contained isoform-specific exons.
Of the combinations that were amplified by PCR, we chose to proceed to assembling the RB and RC isoforms.
We observed frequent, but inconsistent mutations and instances of A to G RNA editing in fragments 3 and 5.
To achieve a final DmCa\textsubscript{v}3 cDNA matching the FlyBase annotation, we reverted one edited site in fragment 3 (\seqsplit{5'-AGTTCAGAGC-3'}) by site-directed mutagenesis.
Since fragment 5 had so many inconsistencies and contained no introns, we used genomic DNA as a template for fragment 5 instead of cDNA.
The final assembled full-length cDNAs were cut with HindIII / XbaI and subcloned into pcDNA3-HE3 downstream of the 5'-UTR from the \emph{Xenopus laevis} $\beta$-globin gene to improve expression in \emph{Xenopus} oocytes.
