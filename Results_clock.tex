\subsection*{Loss of DmCa\textsubscript{v}3 does not affect the core circadian clock}

As in other animals, sleep in \emph{Drosophila} is regulated by the circadian clock, meaning clock mutants generally show altered sleep phenotypes\cite{hendricks:2003aa, parisky:2008aa}.
We therefore asked whether the increased sleep observed in DmCa\textsubscript{v}3-null flies can be attributed to a disrupted circadian clock.
After monitoring their activity over seven days of continuous darkness, we found that \emph{DmCa\textsubscript{v}3\textsuperscript{Gal4}} flies have a slightly elongated period length (24.3 $\pm$ 0.6 vs. 23.9 $\pm$ 0.2), a reduced power of rhythmicity (22.3 $\pm$ 2.9 vs. 53.4 $\pm$ 5.1), and reduced overall percentage of rhythmic flies (70.3\% vs. 92.6\%) when compared to \emph{w\textsuperscript{1118}}.
Still, most \emph{DmCa\textsubscript{v}3\textsuperscript{Gal4}} flies are rhythmic (Fig. \ref{fig:4}a).
The reductions in rhythmicity we observed may be due to an increase in sleep during the transition between subjective day and subjective night.
Consistent with a functional circadian clock, transcriptional oscillation of \emph{period}, one of the core clock genes, is normal in \emph{DmCa\textsubscript{v}3\textsuperscript{Gal4}} flies (Fig. \ref{fig:4}b).
Thus, these results suggest that the increase in sleep caused by loss of DmCa\textsubscript{v}3 cannot be attributed to a direct effect on the molecular clock.

  