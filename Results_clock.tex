\subsection*{Loss of DmCa\textsubscript{v}3 does not affect the core circadian clock}

As in other animals, sleep in \emph{Drosophila} is regulated by the circadian clock, meaning clock mutants generally show altered sleep phenotypes\cite{hendricks:2003aa, parisky:2008aa}.
We therefore asked whether the increased sleep observed in DmCa\textsubscript{v}3-null flies can be attributed to a disruption of the circadian clock.
After monitoring their activity over seven days of continuous darkness, we found that most \emph{DmCa\textsubscript{v}3\textsuperscript{Gal4}} flies have a slightly elongated period length (24.3 $\pm$ 0.6 vs. 23.9 $\pm$ 0.2), a significantly reduced power of rhythmicity (22.3 $\pm$ 2.9 vs. 53.4 $\pm$ 5.1), and a reduced overall percentage of rhythmic flies (70.3\% vs. 92.6\%) when compared to \emph{w\textsuperscript{1118}} (Fig. \ref{fig:4}a).
The circadian phenotype of \emph{DmCa\textsubscript{v}3\textsuperscript{Gal4}} flies is unlikely due to problems in the core circadian clock, as transcriptional oscillation of \emph{period} is normal (Fig. \ref{fig:4}b).
Since the sleep phenotypes of \emph{DmCa\textsubscript{v}3\textsuperscript{Gal4}} flies are already rather subtle, it is difficult to say whether the altered circadian parameters (i.e., the elongated period and reduction in rhythmic power) of \emph{DmCa\textsubscript{v}3\textsuperscript{Gal4}} flies are an independent phenotype or whether they are secondary to the increased sleep during the transition between subjective day and subjective night.
Thus, these results suggest that the increase in sleep caused by loss of DmCa\textsubscript{v}3 cannot be attributed to a direct effect on the molecular clock.
