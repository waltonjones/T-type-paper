\subsection*{Loss of DmCa\textsubscript{v}3 reduces circadian rhythmicity downstream of the core clock}

As in other animals, sleep in \emph{Drosophila} is regulated by the circadian clock, meaning clock mutants generally show altered sleep phenotypes\cite{hendricks:2003aa, parisky:2008aa}.
We therefore asked whether the increased sleep observed in DmCa\textsubscript{v}3-null flies can be attributed to a disruption of the circadian clock.
After monitoring locomotor activity over seven days of continuous darkness, we found that most \emph{DmCa\textsubscript{v}3\textsuperscript{Gal4}} flies have a slightly elongated circadian period length (24.3 $\pm$ 0.6 vs. 23.9 $\pm$ 0.2), a significantly reduced power of rhythmicity (22.3 $\pm$ 2.9 vs. 53.4 $\pm$ 5.1), and a reduced overall percentage of rhythmic flies (70.3\% vs. 92.6\%) when compared to \emph{w\textsuperscript{1118}} controls (Fig. \ref{fig:4}a).
This circadian phenotype of \emph{DmCa\textsubscript{v}3\textsuperscript{Gal4}} flies is unlikely due to problems in the core circadian clock, however, as transcriptional oscillation of \emph{period} is normal (Fig. \ref{fig:4}b).
This means DmCa\textsubscript{v}3 must act downstream of the core circadian clock to affect rhythmic behaviors.
