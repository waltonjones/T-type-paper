\subsection*{Circadian rhythms and sleep homeostasis of Ca-$\alpha$1T-null mutants}

As in other animals, sleep in \emph{Drosophila} is regulated by the circadian clock, meaning clock mutants generally show altered sleep phenotypes\cite{hendricks:2003aa, parisky:2008aa}.
We therefore asked whether the increased sleep observed in Ca-$\alpha$1T-null flies can be attributed to a disruption of the circadian clock.
After monitoring locomotor activity over seven days of continuous darkness, we found that most \emph{Ca-$\alpha$1T\textsuperscript{Gal4}} flies have a slightly elongated circadian period length (24.3 $\pm$ 0.6 vs. 23.9 $\pm$ 0.2), a significantly reduced power of rhythmicity (22.3 $\pm$ 2.9 vs. 53.4 $\pm$ 5.1), and a reduced overall percentage of rhythmic flies (70.3\% vs. 92.6\%) when compared to \emph{w\textsuperscript{1118}} controls (Fig. \ref{fig:4}a).
This circadian phenotype of \emph{Ca-$\alpha$1T\textsuperscript{Gal4}} flies is unlikely due to problems in the core circadian clock, however, as transcriptional oscillation of \emph{period} is normal (Fig. \ref{fig:4}b).
This means Ca-$\alpha$1T must act downstream of the core circadian clock to affect rhythmic behaviors, perhaps affecting the firing of important clock-related neurons.

In addition to being controlled by the circadian clock, sleep is also associated with a homeostatic drive proportional to the time an animal spends awake. Thus, we next examined this homeostatic sleep drive in \emph{Ca-$\alpha$1T\textsuperscript{Gal4}} flies by depriving them of sleep for 24 hours and measuring the resulting sleep rebound. \emph{Ca-$\alpha$1T\textsuperscript{Gal4}} flies do recover slightly more of their lost sleep than \emph{w\textsuperscript{1118}} controls, but the difference is not statistically significant (Fig. \ref{fig:4}c).
