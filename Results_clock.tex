\subsection*{DmCa\textsubscript{v}3 mutant does not affect core clock system}

Sleep is regulated by circadian system and some clock mutants show altered sleep phenotype\cite{12568241, 19038223}.
We examined whether this increased sleep phenotype is due to altered circadian clock.
Activity of \emph{DmCa\textsubscript{v}3\textsuperscript{Gal4}} was monitored for seven days of continuous darkness.
\emph{DmCa\textsubscript{v}3\textsuperscript{Gal4}} flies showed rhythmic activity until the 7th day of continuous dark period although they have increased period length and reduced rhythmicity compared to \emph{w\textsuperscript{1118}} (Fig. \ref{fig:4}a). The reduced percentage of rhythmic flies (90.7\% rhythmicity) and power of rhythmicity (24.0 $\pm$ 3.4) in \emph{DmCa\textsubscript{v}3\textsuperscript{Gal4}} are probably due to the increased sleep at the transition of subjective day and subjective night.
We also checked whether the transcriptional oscillation of \emph{period}, one of the clock genes is normal in \emph{DmCa\textsubscript{v}3\textsuperscript{Gal4}}. 
Period gene in \emph{DmCa\textsubscript{v}3\textsuperscript{Gal4}} showed rhythmic circadian mRNA levels during continuous darkness, peaked at CT12 as emph{w\textsuperscript{1118}} control (Fig. \ref{fig:4}b).
These robust rhythmic behavior in free-running period and rhythmic oscillation of clock gene suggest sleep phenotype of \emph{DmCa\textsubscript{v}3\textsuperscript{Gal4}} is not due to the defect in the circadian system. 
    
  
  
  
  
  