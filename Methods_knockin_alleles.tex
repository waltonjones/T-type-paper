\subsection*{Generation of knock-in alleles}

5' and 3' homologous arms surrounding the \emph{DmCa\textsubscript{v}3} locus were PCR amplified using \emph{w\textsuperscript{1118}} genomic DNA with the following primers: 5'-CGAGATGAATTCTAGCCTCATCAACTGAGC-3'}, \seqsplit{5'-CGAGATGCGGCCGCGAGCAAGCACTAATAGCA-3'}, \seqsplit{5'-GAGATACTAGTCATGCTACAATGTCAGCA-3'}, \seqsplit{5'-CGAGATCTCGAGGGCCACGTATAGGGATGC-3'}.
The homologous arms were then inserted into the \emph{pGX-attP} vector (\href{https://dgrc.bio.indiana.edu/product/View?product=1293}{DGRC \#1293}).
P\{Donor\} flies were generated by P-element based transgenesis of \emph{pGX-attP} containing the homologous arms into the \emph{w\textsuperscript{1118}} genetic background (Genetic Services, Inc., US) and crossed to Flp I-Sce I flies for homologous recombination.
Candidate for proper targeting (i.e., flies with red or mosaic eyes) were selected and verified by PCR.
The \emph{white} marker was removed from a verified strain via Cre-mediated recombination.
The resulting \emph{white}\textsuperscript{-} line was used as a founder (\emph{DmCa\textsubscript{v}3\textsuperscript{Founder,w-}}) for site-specific DNA integration.
\emph{DmCa\textsubscript{v}3\textsuperscript{Gal4}}, \emph{DmCa\textsubscript{v}3\textsuperscript{Rescue}} and \emph{GFP::DmCa\textsubscript{v}3} lines were generated by $\phi$C31 integrase-mediated site-specific integration. 
The Gal4 insert (i.e., splice acceptor-Gal4 CDS-poly A) was amplified from the \emph{pBS-KS-attB1-2-GT-SA-GAL4-Hsp70pA} vector (\href{https://dgrc.bio.indiana.edu/product/View?product=1325}{DGRC \#1325}) with the following primers: \seqsplit{5'-CGTACTCCACGAATTTCTAGAAGTCGATCCAACAT-3'} and \seqsplit{5'-ACCGGCGCGCCTCGACTCTAGAACTAGTGGATCTA-3'}. 
The resulting amplified DNA fragment was sequenced and inserted into the \emph{pGE-attB\textsuperscript{GMR}} vector (\href{https://dgrc.bio.indiana.edu/product/View?product=1295}{DGRC \#1295}) using the EZ-FusionTM cloning kit (Enzynomics, South Korea). 
The Rescue insert was PCR amplified from \emph{w\textsuperscript{1118}} genomic DNA with the following primers: \seqsplit{5'-GCAGAATTCAATCGATTCCATAGATCCGC-3'} and \seqsplit{5'-GCACTCGAGAATTTTGCAACAGGCAGCTA-3'}.
The resulting fragment was inserted into the EcoR I/Xho I site of the \emph{pGE-attB\textsuperscript{GMR}} vector. 
The GFP insert along with a (Gly-Gly-Ser)x4 linker was amplified from the \emph{pBS-KS-attB1-2-PT-SA-SD-1-EGFP-FIAsH-StrepII-TEV-3xFlag} vector (\href{https://dgrc.bio.indiana.edu/product/View?product=1306}{DGRC \#1306}) with the following primers: \seqsplit{5'-GCACCCCAGAAAATGGTGTCCAAGGGCGAGGAGCT-3'} and \seqsplit{5'-CGCTGGCTGTGGCAGGGAACCTCCGCTTCCACCGC-3'}.
The resulting fragment was inserted downstream of the ATG start site in the Rescue construct by inverse PCR (\seqsplit{5'-CTGCCACAGCCAGCGGCAGCG-3'}, \seqsplit{5'-CATTTTCTGGGGTGCCAACTA-3'}) using the 5X In-Fusion HD Enzyme Premix (Clontech).       
\emph{pGE-attB\textsuperscript{GMR}} vectors containing the Gal4, Rescue, and GFP-tagging constructs were injected into \emph{DmCa\textsubscript{v}3\textsuperscript{Founder,w-}} embryos (Rainbow Transgenic Flies, Inc., US) for $\phi$C31-mediated site-specific integration into the \emph{attP} landing site in the DmCa\textsubscript{v}3 locus.
The white-markers of the \emph{DmCa\textsubscript{v}3\textsuperscript{Gal4}} and \emph{DmCa\textsubscript{v}3\textsuperscript{Rescue}} lines were removed before \emph{DmCa\textsubscript{v}3\textsuperscript{Gal4}} and \emph{DmCa\textsubscript{v}3\textsuperscript{Rescue}} were backcrossed to \emph{w\textsuperscript{1118}} for more than 8 generations
After backcrossing, these lines were used for behavioral studies.
