\subsection*{Functional expression of T-type channels in \emph{Xenopus} oocytes}

The information of rat Ca\textsubscript{v}3.1 ($alpha$1G; GenBank accession number AF027984) used as a control was previously reported (Perez-Reyes et al, 1997).
The GenBank accession number of the DmCa\textsubscript{v}3 is NP001096889.
Linearized cDNAs encoding rat Ca\textsubscript{v}3.1 or DmCa\textsubscript{v}3 were used as a template for synthesis of capped cRNAs using T7 RNA polymerase (Ambion, Austin, TX, USA).
The concentrations of cRNAs were estimated based on optical density measured at 260 nm with a spectrophotometer.

Ovary lobes were surgically cut out of female \emph{Xenopus laevis} frogs (Xenopus Express, Haute-Loire, France) which were anesthetized with 0.1\% of 3-aminobenzoic acid ethyl ester.
Removed ovary lobes were torn into small clusters containing ~5 oocytes in a standard oocyte solution (in mM: 100 NaCl, 2 KCl, 1.8 CaCl2, 1 MgCl2, 5 HEPES, 2.5 pyruvic acid, and 50 $\mu$g/ml gentamicin; pH 7.6).
Oocytes were digested by agitating in a Ca\textsubscript{v}3-free solution (in mM: 82.5 NaCl, 2.5 KCl, 1 MgCl2, 5 HEPES; pH 7.6) containing 2 mg/ml collagenase (GIBCO-BRL, Gaithersburg, MD, USA) and trypsin inhibitor (Type III-O, Sigma-Aldrich) for 40~60 minutes to eliminate follicle membranes.
Defolliculated oocytes (stage V-VI) in healthy conditions were manually selected under a dissecting microscope and then incubated in SOS at $18\,^{\circ}\mathrm{C}$ for several hours or overnight for recovery.
Each oocyte was injected with 20-30 ng of DmCa\textsubscript{v}3 cRNA or Ca\textsubscript{v}3.1 cRNA in volume of 50 nl using a Drummond Nanoject pipette injector (Parkway, PA, USA) fixed to a Narishige micromanipulator (Tokyo, Japan) under a stereo-microscope.
    