\subsection*{Pan-neuronal knock-down of Ca-$\alpha$1T increases sleep}

We next asked whether the increased sleep phenotype of \emph{Ca-$\alpha$1T\textsuperscript{Gal4}} flies can be attributed to the function of Ca-$\alpha$1T in the brain.
Pan-neuronal knockdown of Ca-$\alpha$1T (\emph{elav-Gal4$>$UAS-Ca-$\alpha$1T-IR}) increases sleep beyond that of heterozygous controls under both LD and DD conditions (Fig. \ref{fig:5}).
Using the drug-inducible GeneSwitch-Gal4 technique\cite{Osterwalder:2001cl}, we asked whether Ca-$\alpha$1T's influence on sleep occurs during development or whether it is limited to its expression in the adult brain.
Ca-$\alpha$1T knock-down using \emph{elav-GeneSwitch(GS)-Gal4} increases sleep in continuous darkness when compared to non-induced controls (Fig. \ref{fig:S5}).
This suggests the sleep phenotype of Ca-$\alpha$1T-null mutants are unlikely due to developmental defects.

Finally, we sought to narrow down the sleep-regulating role of Ca-$\alpha$1T to a specific brain region or circuit.
We used a range of neuronal Gal4 drivers that cover known sleep centers to knockdown Ca-$\alpha$1T, but none of them were capable of significantly altering sleep (Fig. \ref{fig:6}).
