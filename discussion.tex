\section*{Discussion}

In this study, we first cloned voltage gated T-type Ca\textsuperscript{2+} channel in Drosophila (DmCa\textsubscript{v}3).
Predicted protein size of DmCa\textsubscript{v}3 is about 3200 amino acids, the largest T-type channel cloned to date\cite{senatore:2010aa}.
Sequence similarity of DmCa\textsubscript{v}3 with each mammalian subtypes is ~40\%, though overall biophysical properties of DmCa\textsubscript{v}3 are comparable to those of mammalian T-type calcium channels.
In \emph{Xenopus} oocytes, DmCa\textsubscript{v}3 starts to produce transient LVA currents at around -60 mV similar to mammalian subtypes, which is a hallmark property of T-type Ca\textsuperscript{2+} channels\cite{PerezReyes:2003bw} (Fig. \ref{fig:1}A).
This suggests fly T-type calcium channel shares structural determinants for gating with its mammalian homologs despite of substantial divergence between the channels.
Another feature that shared among mammalian T-type calcium channels is existence of small fraction of channels remain open in the overlapping voltage ranges of activation and steady-state inactivation curves.
Window currents through these channels are thought to regulate intracellular calcium concentration and signal amplification\cite{PerezReyes:2003bw,crunelli:2004aa,dreyfus:2010aa}.
We found DmCa\textsubscript{v}3 has apparently larger window current than rat Ca\textsubscript{v}3.1 because of positive shift of steady-state inactivation and negative shift of activation curve (Fig. \ref{fig:1}B).
In neurons with typical resting membrane potential within the window region, DmCa\textsubscript{v}3 may affect resting calcium influx and have similar functions with mammalian homologs.

While three T-type calcium channel subtypes in mammals share overall structural and functional identity, several biophysical features discriminate among T-type channel subtypes.
Previous study revealed voltage-dependent activation/inactivation kinetics of Ca\textsubscript{v}3.1 and Ca\textsubscript{v}3.2 are the fastest, while Ca\textsubscript{v}3.3 has strikingly slower kinetics producing less transient calcium currents than other subtypes\cite{lee:1999aa,klockner:1999aa}.
We found DmCa\textsubscript{v}3 show somewhat intermediate kinetics between Ca\textsubscript{v}3.1/ Ca\textsubscript{v}3.2 and Ca\textsubscript{v}3.3 (Fig. \ref{fig:1}C).
This does not seem to be a common feature of invertebrate T-type channels though, because recombinant T-type calcium channel of \emph{Lymnaea stagnalis} (LCa\textsubscript{v}3) showed similar kinetics to Ca\textsubscript{v}3.2\cite{senatore:2010aa}.
Another criteria that distinguish subtypes is its permeability to different divalent cations\cite{mcrory:2000aa,shcheglovitov:2007aa}.
We showed DmCa\textsubscript{v}3 produces larger currents with Ba\textsuperscript{2+} as charge carrier than Ca\textsubscript{v}3 similar to Ca\textsubscript{v}3.2 or Ca\textsubscript{v}3.3, while Ca\textsubscript{v}3.1 has opposite permeability ratio (1D and 1E).
LCa\textsubscript{v}3 channel also produce larger currents with Ba\textsuperscript{2+} than Ca\textsubscript{v}3\cite{senatore:2010aa}.
Sensitivity to blockage by Ni\textsuperscript{2+} also discriminate among subtype channels.
Ni\textsuperscript{2+} sensitivity of DmCa\textsubscript{v}3 is comparable to Ca\textsubscript{v}3.2, the most sensitive subtype (Fig. \ref{fig:1}F) whereas LCa\textsubscript{v}3 has much less sensitivity to Ni\textsuperscript{2+} similar to Ca\textsubscript{v}3.1 or Ca\textsubscript{v}3.3\cite{senatore:2010aa}.
These results suggest invertebrate T-type channels have unique structures not absolutely identical to one of three mammalian subtypes, while key structural determinants for producing a \"T\" current is well conserved.

Using GFP-tagging protein expression, we determined expression patterns of fly T-type Ca\textsuperscript{2+} channel.
Similar to mammalian T-type Ca\textsuperscript{2+} channels which show widely distributed throughout the brain\cite{talley:1999aa}, DmCa\textsubscript{v}3 is expressed in various region of adult brain.
This implies T-type channel may have various function in information processing.
In addition, while mammalian T-type Ca\textsuperscript{2+} channels are known to exit at dendritic areas, DmCa\textsubscript{v}3 is localized in soma, dendritic, and axon terminal areas (Fig. ), suggesting a versatile role compared to mammalian T-type Ca\textsuperscript{2+} channels. 

It is surprising that DmCa\textsubscript{v}3 null mutants are homozygote viable with normal appearance and have no movement defects.
A previous study suggests that DmCa\textsubscript{v}2, a homolog of HVA Ca\textsubscript{v}2 subfamily, also play a role in the generation of low voltage-activated Ca\textsubscript{v}3 currents\cite{Ryglewski:2012jk}.
It is plausible that other HVA channels compensate the role of DmCa\textsubscript{v}3 during development.
However, flies lacking DmCa\textsubscript{v}3 show a clear sleep phenotype, suggesting a specific function of this gene in the regulation of sleep/wake behavior.

In mammalians, T-type Ca\textsuperscript{2+} channel play a relevant role in sleep consolidation by generating sleep oscillations during NREM sleep.
In response to inhibitory inputs, T-type Ca\textsuperscript{2+} channels mediates the generation of low-threshold burst spikes which have a strong post-synaptic impact.
In the thalamus, excitatory thalamocortical relay (TC) neurons and GABAergic nRT neurons express Ca\textsubscript{v}3.1 and Ca\textsubscript{v}3.2/Ca\textsubscript{v}3.3, respectively, and these neurons are connected reciprocally.
During sleep, hyperpolarization of TC neurons generates low-threshold burst spikes attributing to activation of Ca\textsubscript{v}3.1, which stimulate nRT neurons to generate low-threshold bust spike depending on activation of Ca\textsubscript{v}3.2/Ca\textsubscript{v}3.3, which then activate TC neurons again.
This reverberatory interaction mediates the generation of sleep oscillations in the thalamocortical pathways and contribute to the consolidation of sleep state.
Supporting this theory, mice lacking Ca\textsubscript{v}3.1 showed an instability in the maintenance of sleep duration\cite{Lee:2004ey}.

In contrast, DmCa\textsubscript{v}3 plays an opposite role to that reported in mammals.
Increased sleep need in DmCa\textsubscript{v}3 mutants suggests T-type Ca\textsuperscript{2+} channel play wake-promoting role.
Circadian rhythm of DmCa\textsubscript{v}3 mutant showed weak, but significant rhythmicity during constant darkness and core clock gene period showed normal transcriptional oscillation in DmCa\textsubscript{v}3 mutant, suggesting that T-type Ca\textsuperscript{2+} channel exert the wake-promoting function independently from molecular clock system.

Studies have suggested that T-type Ca\textsuperscript{2+} channels are able to be associated with the wake-promoting mechanism as following.
In wake-up call hypothesis, the low-threshold burst spikes depending on activation of T-type Ca\textsuperscript{2+} channel has a strong post-synaptic impact and thereby stimulating the cortex efficiently\cite{swadlow:2001aa}, which may help the sensation or attention to specific sensory stimuli at the cortex.
Another study showed that T-type channel inhibition with specific blocker TTA-A2 decreased wake and increased delta sleep, but not in Ca\textsubscript{v}3.1/Ca\textsubscript{v}3.3 double knockout mice\cite{kraus:2010aa}, suggesting that the role of T-type calcium channel in sleep regulation may complex.
However, the wake-promoting role of low-threshold burst spikes have been controversial\cite{steriade:2001aa}. 

Finally, our study strongly suggest that the excitatory currents induced by that T-type Ca\textsuperscript{2+} channels play a role in enhancing wakefulness or arousal in invertebrate.
This wake-promoting function and lack of deep sleep in invertebrate seems to be relevant for surveillance against predation.
Invertebrates may lack deep sleep state such as non-rapid eye sleep (NREM) because they lack the sleep-promoting mechanism that depends on T-type Ca\textsuperscript{2+} channel.
This species-dependent differences of sleep regulation may give insights on the evolution of sleep pattern differentiation and the mechanism.