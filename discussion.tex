\section*{Discussion}

In this study, we cloned the only voltage-gated T-type Ca\textsuperscript{2+} channel from \emph{Drosophila}, \emph{DmCa\textsubscript{v}3}.
DmCa\textsubscript{v}3 is the largest T-type channel cloned to date, measuring 3205 amino acids\cite{senatore:2010aa}.
Electrophysiological characterization of DmCa\textsubscript{v}3 in \emph{Xenopus} oocytes showed that DmCa\textsubscript{v}3 has all the hallmark properties of a T-type channel: low-threshold activation at around \textminus60 mV, a maximal current output at \textminus20 mV, transient current kinetics elicited by a step-pulse protocol producing a ``criss-crossing'' pattern, and slow deactivation of tail currents (Fig. \ref{fig:1}).
These biophysical properties are also consistent with previous studies that implicated DmCa\textsubscript{v}3 in low-voltage-activated (LVA) currents in both the central and peripheral nervous systems of the fly\cite{Ryglewski:2012jk, Iniguez:2013ib}.

Mammalian genomes contain three T-type Ca\textsuperscript{2+}channel genes (i.e., Ca\textsubscript{v}3.1-3.3), while the fly genome contains only one.
We therefore measured DmCa\textsubscript{v}3 for some of the characteristics that distinguish the three mammalian channels.
In terms of current kinetics, DmCa\textsubscript{v}3 is more similar to mammalian Ca\textsubscript{v}3.1 and Ca\textsubscript{v}3.2 than Ca\textsubscript{v}3.3, which exhibits considerably  slower kinetics.
In terms of both its relative permeability to Ba\textsuperscript{2+} over Ca\textsuperscript{2+} and its sensitivity to nickel inhibition, DmCa\textsubscript{v}3 is most similar to Ca\textsubscript{v}3.2\cite{kang:2006aa, park:2013aa}.

The three mammalian T-type Ca\textsuperscript{2+} channels,  each with their own distinct biophysical properties, are expressed in largely complementary patterns of neurons throughout the brain, conferring considerable functional diversity. Areas of particularly strong expression include those important for the gating and processing of sensory inputs, motor control, learning and memory, as well as reward circuits\cite{talley:1999aa}. 
Using a GFP-tagged knock-in allele, we report in this study that DmCa\textsubscript{v}3 is expressed broadly across the adult fly brain in structures reminiscent of the mammalian T-type Ca\textsuperscript{2+} channels.
These include sensory neuropils (i.e., the optic and antennal lobes, the antennal mechanosensory and motor centers, the anterior ventrolateral protocerebrum, and the subesophageal zone), motor-associated neuropils (i.e., the central complex), and those associated with learning, memory, and reward (i.e., the mushroom bodies).
It is still unclear, however, whether the different isoforms predicted to originate from the \emph{DmCa\textsubscript{v}3} locus will have different biophysical properties or different distributions around the brain. 

The relatively subtle phenotypes of the homozygous viable Ca\textsubscript{v}3 mutant mice are often ascribed to functional compensation among the various Ca\textsubscript{v}3.1-3 isoforms. \textcolor{red}{Maybe a citation here?}
It is perhaps surprising, then, that despite its broad and relatively strong expression across adult fly brains, null mutants of the only fly T-type channel, DmCa\textsubscript{v}3, are also homozygous viable and lack any overt phenotypes.
A previous study suggests that DmCa\textsubscript{v}2, a homologue of HVA Ca\textsubscript{v}2 subfamily, also play a role in the generation of low voltage-activated Ca\textsuperscript{2+} currents\cite{Ryglewski:2012jk}.
It is plausible that other HVA channels compensate the role of DmCa\textsubscript{v}3 during development.
However, flies lacking DmCa\textsubscript{v}3 show a clear sleep phenotype, suggesting a specific function of this gene in the regulation of sleep/wake behavior.
In mammals, T-type Ca\textsuperscript{2+} channels play a relevant role in sleep consolidation by generating sleep oscillations during NREM sleep.
In response to inhibitory inputs, T-type Ca\textsuperscript{2+} channels mediates the generation of low-threshold burst spikes which have a strong post-synaptic impact.
In the thalamus, excitatory thalamocortical relay (TC) neurons and GABAergic nRT neurons express Ca\textsubscript{v}3.1 and Ca\textsubscript{v}3.2/Ca\textsubscript{v}3.3 respectively, and these neurons are connected reciprocally.
During sleep, hyperpolarization of TC neurons generates low-threshold burst spikes attributing to activation of Ca\textsubscript{v}3.1, which stimulate nRT neurons to generate low-threshold bust spike depending on activation of Ca\textsubscript{v}3.2/Ca\textsubscript{v}3.3, which then activate TC neurons again.
This reverberatory interaction mediates the generation of sleep oscillations in the thalamocortical pathways and contribute to the consolidation of sleep state.
Supporting this theory, mice lacking Ca\textsubscript{v}3.1 showed an instability in the maintenance of sleep duration\cite{Lee:2004ey}.

On the other hand, some studies have suggested that T-type Ca\textsuperscript{2+} channels are able to be associated with the wake-promoting mechanism as following.
In wake-up call hypothesis, the low-threshold burst spikes depending on activation of T-type Ca\textsuperscript{2+} channel has a strong post-synaptic impact and thereby stimulating the cortex efficiently\cite{swadlow:2001aa}, which may help the sensation or attention to specific sensory stimuli at the cortex.
Another study showed that T-type channel inhibition with specific blocker TTA-A2 decreased wake and increased delta sleep, but not in Ca\textsubscript{v}3.1/Ca\textsubscript{v}3.3 double knockout mice\cite{kraus:2010aa}.
These controversial and complex results suggest that the role of mammalian T-type calcium channels in sleep regulation may complex. 

In this study, we clearly showed DmCa\textsubscript{v}3 plays negative role in sleep regulation.
Increased sleep need in DmCa\textsubscript{v}3 mutants suggests T-type Ca\textsuperscript{2+} channel play wake-promoting role.
Circadian rhythm of DmCa\textsubscript{v}3 mutant showed weak, but significant rhythmicity during constant darkness and core clock gene period showed normal transcriptional oscillation in DmCa\textsubscript{v}3 mutant, suggesting that T-type Ca\textsuperscript{2+} channel exert the wake-promoting function independently from molecular clock system.

This wake-promoting function and lack of deep sleep in invertebrate seems to be relevant for surveillance against predation.
Invertebrates may lack deep sleep state such as non-rapid eye sleep (NREM) because they lack the sleep-promoting mechanism that depends on T-type Ca\textsuperscript{2+} channel.
This species-dependent differences of sleep regulation may give insights on the evolution of sleep pattern differentiation and the mechanism.
  
  
  
  
  
  
  
  