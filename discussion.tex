\section*{Discussion}

In this study, we cloned voltage gated T-type Ca\textsuperscript{2+} channel in Drosophila (DmCa\textsubscript{v}3). The protein size of DmCa\textsubscript{v}3 is 3205 amino acids, the largest T-type channel cloned to date\cite{20056611}. Electrophysiological characterization of DmCa\textsubscript{v}3 showed that the channel has the hallmark properties of native T-type channels described from isolated cells as well as those of cloned T-type channel; 1) low-threshold activation at around \textminus60 mV and creation of maximal current amplitude at \textminus20 mV in the I-V plot, 2) transient current kinetics during step pulses producing a criss-crossing pattern between current traces elicited by an I-V protocol and 3) slow deactivation of tail currents (Fig. \ref{fig:1}). These results indicate that DmCa\textsubscript{v}3 is a T-type channel cloned from \emph{Drosophila melanogaster}. Biophysical properties of the cloned channel are consistent with previous studies showing that DmCa\textsubscript{v}3 mediates low-voltage-activated (LVA) currents both in central and peripheral nervous system\cite{22183725, 23864373}. Analysis of the fly genome suggests that DmCa\textsubscript{v}3 is unique T-type channel isoform in the fly, which is different from existence of the three isoforms (Ca\textsubscript{v}3.1-3.3) in mammals. In current kinetics of view, DmCa\textsubscript{v}3 is similar to mammalian Ca\textsubscript{v}3.1 and Ca\textsubscript{v}3.2 rather than mammalian Ca\textsubscript{v}3.3 with much slower current kinetics. In terms of relative permeability and nickel inhibition sensitivity, the preferential permeation of Ba\textsuperscript{2+} over Ca\textsuperscript{2+} and nickel sensitive inhibition of DmCa\textsubscript{v}3 suggest that the fly T-type channel resembles to mammalian Ca\textsubscript{v}3.2\cite{16377633, 23849427}.

Broad expression patterns of DmCa\textsubscript{v}3 in various region of adult brain are reminiscent of the mammalian T-type Ca\textsuperscript{2+} channels widely distributed throughout the brain\cite{talley:1999aa}. 
Three subtypes of mammalian T-type Ca\textsuperscript{2+} channels are express in largely complementary manner with distinct biophysical properties, which confers the functional diversity. 
Whether transcriptional isoforms of DmCa\textsubscript{v}3 have different biophysical properties or distribution in the brain is not clear.    
It is surprising that despite of broad and strong expression of DmCa\textsubscript{v}3, null mutants are homozygote viable with normal appearance and have no movement defects.
A previous study suggests that DmCa\textsubscript{v}2, a homologue of HVA Ca\textsubscript{v}2 subfamily, also play a role in the generation of low voltage-activated Ca\textsubscript{v}3 currents\cite{Ryglewski:2012jk}.
It is plausible that other HVA channels compensate the role of DmCa\textsubscript{v}3 during development.
However, flies lacking DmCa\textsubscript{v}3 show a clear sleep phenotype, suggesting a specific function of this gene in the regulation of sleep/wake behavior.
In mammals, T-type Ca\textsuperscript{2+} channels play a relevant role in sleep consolidation by generating sleep oscillations during NREM sleep.
In response to inhibitory inputs, T-type Ca\textsuperscript{2+} channels mediates the generation of low-threshold burst spikes which have a strong post-synaptic impact.
In the thalamus, excitatory thalamocortical relay (TC) neurons and GABAergic nRT neurons express Ca\textsubscript{v}3.1 and Ca\textsubscript{v}3.2/Ca\textsubscript{v}3.3 respectively, and these neurons are connected reciprocally.
During sleep, hyperpolarization of TC neurons generates low-threshold burst spikes attributing to activation of Ca\textsubscript{v}3.1, which stimulate nRT neurons to generate low-threshold bust spike depending on activation of Ca\textsubscript{v}3.2/Ca\textsubscript{v}3.3, which then activate TC neurons again.
This reverberatory interaction mediates the generation of sleep oscillations in the thalamocortical pathways and contribute to the consolidation of sleep state.
Supporting this theory, mice lacking Ca\textsubscript{v}3.1 showed an instability in the maintenance of sleep duration\cite{Lee:2004ey}.

On the other hand, some studies have suggested that T-type Ca\textsuperscript{2+} channels are able to be associated with the wake-promoting mechanism as following.
In wake-up call hypothesis, the low-threshold burst spikes depending on activation of T-type Ca\textsuperscript{2+} channel has a strong post-synaptic impact and thereby stimulating the cortex efficiently\cite{swadlow:2001aa}, which may help the sensation or attention to specific sensory stimuli at the cortex.
Another study showed that T-type channel inhibition with specific blocker TTA-A2 decreased wake and increased delta sleep, but not in Ca\textsubscript{v}3.1/Ca\textsubscript{v}3.3 double knockout mice\cite{kraus:2010aa}. These controversial and complex results suggest that the role of mammalian T-type calcium channel in sleep regulation may complex. 

In this study, we clearly showed DmCa\textsubscript{v}3 plays negative role in sleep regulation.
Increased sleep need in DmCa\textsubscript{v}3 mutants suggests T-type Ca\textsuperscript{2+} channel play wake-promoting role.
Circadian rhythm of DmCa\textsubscript{v}3 mutant showed weak, but significant rhythmicity during constant darkness and core clock gene period showed normal transcriptional oscillation in DmCa\textsubscript{v}3 mutant, suggesting that T-type Ca\textsuperscript{2+} channel exert the wake-promoting function independently from molecular clock system.

This wake-promoting function and lack of deep sleep in invertebrate seems to be relevant for surveillance against predation.
Invertebrates may lack deep sleep state such as non-rapid eye sleep (NREM) because they lack the sleep-promoting mechanism that depends on T-type Ca\textsuperscript{2+} channel.
This species-dependent differences of sleep regulation may give insights on the evolution of sleep pattern differentiation and the mechanism.
  
  
  
  