\section*{Discussion}

In this study, we cloned the only voltage-gated T-type Ca\textsuperscript{2+} channel from \emph{Drosophila}, \emph{DmCa\textsubscript{v}3}.
DmCa\textsubscript{v}3 is the largest T-type channel cloned to date, measuring 3205 amino acids\cite{senatore:2010aa}.
Electrophysiological characterization of DmCa\textsubscript{v}3 in \emph{Xenopus} oocytes showed that DmCa\textsubscript{v}3 has all the hallmark properties of a T-type channel: low-threshold activation at around \textminus60 mV, a maximal current output at \textminus20 mV, transient current kinetics elicited by a step-pulse protocol producing a ``criss-crossing'' pattern, and slow deactivation of tail currents (Fig. \ref{fig:1}).
These biophysical properties are also consistent with previous studies that implicated DmCa\textsubscript{v}3 in low-voltage-activated (LVA) currents in both the central and peripheral nervous systems of the fly\cite{Ryglewski:2012jk, Iniguez:2013ib}.

Mammalian genomes contain three T-type Ca\textsuperscript{2+}channel genes (i.e., Ca\textsubscript{v}3.1-3.3), while the fly genome contains only one.
We therefore measured DmCa\textsubscript{v}3 for some of the characteristics that distinguish the three mammalian channels.
In terms of current kinetics, DmCa\textsubscript{v}3 is more similar to mammalian Ca\textsubscript{v}3.1 and Ca\textsubscript{v}3.2 than Ca\textsubscript{v}3.3, which exhibits considerably  slower kinetics.
In terms of both its relative permeability to Ba\textsuperscript{2+} over Ca\textsuperscript{2+} and its sensitivity to nickel inhibition, DmCa\textsubscript{v}3 is most similar to Ca\textsubscript{v}3.2\cite{kang:2006aa, park:2013aa}.

The three mammalian T-type Ca\textsuperscript{2+} channels,  each with their own distinct biophysical properties, are expressed in largely complementary patterns of neurons throughout the brain, conferring considerable functional diversity. Areas of particularly strong expression include those important for the gating and processing of sensory inputs, motor control, learning and memory, as well as reward circuits\cite{talley:1999aa}. 
Using a GFP-tagged knock-in allele, we report in this study that DmCa\textsubscript{v}3 is expressed broadly across the adult fly brain in structures reminiscent of the mammalian T-type Ca\textsuperscript{2+} channels.
These include sensory neuropils (i.e., the optic and antennal lobes, the antennal mechanosensory and motor centers, the anterior ventrolateral protocerebrum, and the subesophageal zone), motor-associated neuropils (i.e., the central complex), and those associated with learning, memory, and reward (i.e., the mushroom bodies).
It is still unclear, however, whether the different isoforms predicted to originate from the \emph{DmCa\textsubscript{v}3} locus will have different biophysical properties or different distributions around the brain. 

Considering their broad expression, T-type knockout mice appear remarkably healthy and subtle mutant phenotypes emerge only upon close inspection.
Sleep, in particular, has emerged as a focal point in the search for a physiological function for the T-type channels.
Mammalian T-type Ca\textsuperscript{2+} channels may act as sleep stabilizers and may help generate the burst firing necessary for the sleep oscillations of deep NREM sleep.
Unfortunately, the three separate mammalian T-type genes all undergo alternative splicing to produce various channel isoforms that each have specific biophysical properties, neuroanatomical and subcellular localizations, and varying abilities to interact with other ion channels.
All these variables combine to make it difficult if not impossible to define a precise physiological role in sleep for T-type channels as a group.
Although Ca\textsubscript{v}3.1 knockout mice lack the delta oscillations characteristic of deep sleep and show reduced total sleep\cite{Lee:2004ey}, when the knockout is limited to the rostral midline thalamus, sleep is still reduced, but delta waves are mildly increased\cite{anderson:2005aa}. Another more recent study showed that treatment with the T-type-specific channel blocker TTA-A2 enhances sleep and delta rhythms in wild type mice but not Ca\textsubscript{v}3.1/Ca\textsubscript{v}3.3 double knockout mice\cite{kraus:2010aa}. In other words, manipulation of T-type channels can both enhance and reduce total sleep and deep delta-wave sleep depending on the experimental context.

Although perhaps disrespectful of the actual complexity of the situation, the subtle phenotypes of the homozygous viable Ca\textsubscript{v}3 mutant mice are often ascribed to functional compensation among the various Ca\textsubscript{v}3.1-3 isoforms. \textcolor{red}{Maybe a citation here?}
It is surprising, then, that despite its broad and relatively strong expression across adult fly brains, null mutants of the one and only fly T-type channel, DmCa\textsubscript{v}3, are also homozygous viable and lack any overt phenotypes.
We do show in this study, however, that DmCa\textsubscript{v}3-null mutants sleep more than controls.
We have been unable so far to assign this phenotype to a specific neuronal subpopulation, but our results suggest an overall wake-promoting role for DmCa\textsubscript{v}3.
This wake-promoting function also seems to be independent of the circadian clock, as DmCa\textsubscript{v}3-null mutants show weak but significant rhythmicity in constant darkness and normal oscillation of the core clock gene \emph{period}.

Judging from the complexity of T-type physiology so far reported in mice, it will be interesting to see whether future studies focused on the technically demanding study of isoform-specific expression patterns and isoform-specific rescues in both mice and flies will clarify how T-type channels can at various times both enhance and reduce sleep.

  