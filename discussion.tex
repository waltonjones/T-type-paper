\section*{Discussion}

In this study, we cloned the only voltage-gated T-type Ca\textsuperscript{2+} channel from \emph{Drosophila}, DmCa\textsubscript{v}3.
DmCa\textsubscript{v}3 is the largest T-type channel cloned to date, measuring 3205 amino acids\cite{senatore:2010aa}.
Electrophysiological characterization of DmCa\textsubscript{v}3 in \emph{Xenopus} oocytes showed that DmCa\textsubscript{v}3 has the hallmark properties of a T-type channel: low-threshold activation at around $-$60 mV, a maximal current output at $-$20 mV, transient current kinetics elicited by a step-pulse protocol producing a ``criss-crossing'' pattern, and slow deactivation of tail currents (Fig. \ref{fig:1}).
These biophysical properties are also consistent with previous studies that implicated DmCa\textsubscript{v}3 in low-voltage-activated (LVA) currents in both the central and peripheral nervous systems of the fly\cite{Ryglewski:2012jk, Iniguez:2013ib}.

Mammalian genomes contain three T-type Ca\textsuperscript{2+}channel genes (i.e., Ca\textsubscript{v}3.1-3.3), while the fly genome contains only one.
We therefore measured DmCa\textsubscript{v}3 for some of the characteristics that distinguish the three mammalian channels.
In terms of current kinetics, DmCa\textsubscript{v}3 is more similar to mammalian Ca\textsubscript{v}3.1 and Ca\textsubscript{v}3.2 than Ca\textsubscript{v}3.3, which exhibits considerably  slower kinetics.
In terms of both its relative permeability to Ba\textsuperscript{2+} over Ca\textsuperscript{2+} and its sensitivity to nickel inhibition, DmCa\textsubscript{v}3 is most similar to Ca\textsubscript{v}3.2\cite{kang:2006aa, park:2013aa}.

The three mammalian T-type Ca\textsuperscript{2+} channels,  each with their own distinct biophysical properties, are expressed in largely complementary patterns of neurons throughout the brain, conferring considerable functional diversity. Areas of particularly strong expression include those important for the gating and processing of sensory inputs, motor control, learning and memory, as well as reward circuits\cite{talley:1999aa}. 
Using a GFP-tagged knock-in allele, we report in this study that DmCa\textsubscript{v}3 is expressed broadly across the adult fly brain in structures reminiscent of the mammalian T-type Ca\textsuperscript{2+} channels.
These include sensory neuropils (i.e., the optic and antennal lobes, the antennal mechanosensory and motor centers, the anterior ventrolateral protocerebrum, and the subesophageal zone), motor-associated neuropils (i.e., the central complex), and those associated with learning, memory, and reward (i.e., the mushroom bodies).
It is still unclear, however, whether the different isoforms predicted to originate from the \emph{DmCa\textsubscript{v}3} locus will have different biophysical properties or different distributions around the brain. 

Considering their broad expression, T-type knockout mice appear healthy and subtle mutant phenotypes emerge only upon close inspection.
Sleep, in particular, has emerged as a focal point in the search for a physiological function for the T-type channels.
Mammalian T-type Ca\textsuperscript{2+} channels may act as sleep stabilizers and may help generate the burst firing necessary for the sleep oscillations of deep NREM sleep.
Unfortunately, the three separate mammalian T-type genes all undergo alternative splicing to produce various channel isoforms that each have specific biophysical properties, neuroanatomical and subcellular localizations, and varying abilities to interact with other ion channels.
All these variables and more combine to make it difficult if not impossible to define a precise physiological role in sleep for T-type channels as a group.
Although Ca\textsubscript{v}3.1 knockout mice lack the delta oscillations characteristic of deep sleep and show reduced total sleep\cite{Lee:2004ey}, when the knockout is limited to the rostral midline thalamus, sleep is still reduced, but delta waves are mildly increased\cite{anderson:2005aa}. Another more recent study showed that treatment with the T-type-specific channel blocker TTA-A2 enhances sleep and delta rhythms in wild type mice but not Ca\textsubscript{v}3.1/Ca\textsubscript{v}3.3 double knockout mice\cite{kraus:2010aa}. In other words, manipulation of T-type channels can both enhance and reduce total sleep and deep delta-wave sleep depending on the experimental context.

Although perhaps an underestimate of the actual complexity of the situation, the subtlety of the phenotypes of the homozygous viable Ca\textsubscript{v}3 mutant mice are often ascribed to functional compensation among the various Ca\textsubscript{v}3.1-3 isoforms\cite{senatore:2012aa}.
We, therefore, expected that a behavioral investigation of the one and only fly T-type channel, DmCa\textsubscript{v}3, would uncover less subtle sleep phenotypes.
We were thus surprised to find, that despite its broad and relatively strong expression across adult fly brains, DmCa\textsubscript{v}3-null mutants, like the Ca\textsubscript{v}3.1-null mice, are homozygous viable and lack any overt phenotypes.
Upon closer examination, however, we observed that DmCa\textsubscript{v}3-null mutants sleep more than controls, especially in constant darkness.
In addition, DmCa\textsubscript{v}3-null mutants also have a circadian phenotype: an elongated circadian period and a reduction in rhythmic power.
It is difficult to say, though, whether these altered circadian parameters observed in DmCa\textsubscript{v}3-null mutants are an independent phenotype or whether they are secondary to the sleep phenotype.
Rhythmic power is proportional to the magnitude of the changes in activity level and the regularity with which they occur.
Since the increased sleep observed in the DmCa\textsubscript{v}3-null mutants does reduce the change in overall activity level between subjective day and subjective night, the increased sleep must also cause a reduction in rhymic power.
Although we were able to replicate the increased sleep phenotype of DmCa\textsubscript{v}3-null mutants via pan-neuronal knock-down of DmCa\textsubscript{v}3, we were unable to further narrow the cause of this phenotype to a specific neuronal subpopulation.
This was in spite of making many attempts with a host of neuronal Gal4 driver lines ranging from broadly expressed enhancer traps and neurotransmitter Gal4 drivers to much more narrowly expressed neuropeptide drivers.
Still, our results suggest an overall wake-promoting role for DmCa\textsubscript{v}3.

The ``three channel'' compensation hypothesis in mice may yet turn out to be correct, but our results in flies suggest that other factors---isoform-specific differences, differences related to protein--protein interactions, or even something completely unforeseen---may allow mice and flies lacking these broadly expressed and highly conserved ion channels to still function remarkably well.
It will be interesting to see whether future studies focused on the technically demanding study of isoform-specific expression patterns and isoform-specific rescues in both mice and flies will clarify how T-type channels can at various times and in various contexts both enhance and reduce sleep.
