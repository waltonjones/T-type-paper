\subsection*{DmCa\textsubscript{v}3 mutants show increased sleep}

Since the Gal4 coding sequence inserted into \emph{DmCa\textsubscript{v}3\textsuperscript{Founder}} to produce the \emph{DmCa\textsubscript{v}3\textsuperscript{Gal4}} allele included a termination sequence (Fig. \ref{fig:3}a), \emph{DmCa\textsubscript{v}3\textsuperscript{Gal4}} is likely a null allele.
As expected, we were unable to detect DmCa\textsubscript{v}3 expression in the fly head lysates from \emph{DmCa\textsubscript{v}3\textsuperscript{Gal4}} in western blot analyses using polyclonal DmCa\textsubscript{v}3-specific antisera (Fig. \ref{fig:3}a). We did, however, detect strong DmCa\textsubscript{v}3 expression in lysates from \emph{w\textsuperscript{1118}} controls and from a \emph{DmCa\textsubscript{v}3\textsuperscript{Rescue}} allele in which the fragment deleted in both the \emph{DmCa\textsubscript{v}3\textsuperscript{Founder}} and \emph{DmCa\textsubscript{v}3\textsuperscript{Gal4}} alleles was re-inserted (Fig. \ref{fig:3}a and b).
\emph{DmCa\textsubscript{v}3\textsuperscript{Gal4}} homozygotes are viable and fertile with normal appearance and no obvious movement defects.
Two of the mammalian T-type channel subtypes, Ca\textsubscript{v}3.1 and Ca\textsubscript{v}3.3, have been implicated in the generation of the neural oscillations characteristic of NREM sleep.
Flies have a well-established sleep-like state that shares some features with mammalian sleep, but it remains unclear whether flies have a stage akin to mammalian NREM sleep.
Still, we hypothesized that DmCa\textsubscript{v}3-null flies may exhibit sleep defects.

\emph{DmCa\textsubscript{v}3\textsuperscript{Gal4}} flies show increased total sleep under both 12h:12h light-dark (LD) and constant dark (DD) conditions, but this phenotype is particularly prominent during the subjective day under continuous dark (DD) conditions (Fig. \ref{fig:3}c and d).
Total sleep time is rescued, however, in \emph{DmCa\textsubscript{v}3\textsuperscript{Rescue}} flies to levels similar to those of \emph{w\textsuperscript{1118}} control flies (Fig. \ref{fig:3}c and d).
By measuring waking locomotor activity, we were able to confirm that the increased sleep of \emph{DmCa\textsubscript{v}3\textsuperscript{Gal4}} flies is not an artifact of a generalized reduction in movement. In fact, \emph{DmCa\textsubscript{v}3\textsuperscript{Gal4}} show slightly higher levels of waking activity than their respective controls (Fig. \ref{fig:3}e).

Normal fly sleep consists of a number of sleep bouts.
We, therefore, asked whether the increased sleep of \emph{DmCa\textsubscript{v}3\textsuperscript{Gal4}} flies is a result of an increased number of sleep bouts, prolonged bout duration, or both.
\emph{DmCa\textsubscript{v}3\textsuperscript{Gal4}} flies do show reduced sleep bout number under LD conditions, but this phenotype is not rescued in \emph{DmCa\textsubscript{v}3\textsuperscript{Rescue}} flies (Fig. \ref{fig:3}f).
Sleep bout length, on the other hand, is increased under both LD and DD conditions and rescued in \emph{DmCa\textsubscript{v}3\textsuperscript{Rescue}} flies (Fig. \ref{fig:3}g).

To confirm that this elevated sleep phenotype is specific to DmCa\textsubscript{v}3 loss-of-function, we generated three independent deletion mutants via imprecise P-element excision.
As expected, all three deletion mutants as well as a trans-heterozygous mutants ($\Delta$3/$\Delta$115) show increased sleep, especially in constant darkness (Fig. \ref{fig:S2}).
In addition, knockdown of DmCa\textsubscript{v}3 in its own neurons (\emph{DmCa\textsubscript{v}3\textsuperscript{Gal4}>UAS-DmCa\textsubscript{v}3-IR} driver increases sleep after the third day of continuous dark (Fig. \ref{fig:S3}).
Together, these results implicate DmCa\textsubscript{v}3 as a novel sleep-inhibitor of fly sleep.
