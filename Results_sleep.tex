\subsection*{DmCa\textsubscript{v}3 mutants show increased sleep}

Since the Gal4 coding sequence inserted into DmCa\textsubscript{v}3\textsuperscript{Founder} to produce the DmCa\textsubscript{v}3\textsuperscript{Gal4} allele included a termination sequence (Fig. \ref{fig:S1}C), DmCa\textsubscript{v}3\textsuperscript{Gal4} is likely a null allele.
As expected, we were unable to detect DmCa\textsubscript{v}3 expression in whole fly lysates DmCa\textsubscript{v}3\textsuperscript{Gal4} flies in western blot analyses using polyclonal DmCa\textsubscript{v}3-specific antisera. We did, however, detect strong DmCa\textsubscript{v}3 expression in fly lysates from w\textsuperscript{1118} controls and from a DmCa\textsubscript{v}3\textsuperscript{Rescue} allele in which the fragment deleted in both the DmCa\textsubscript{v}3\textsuperscript{Founder} and DmCa\textsubscript{v}3\textsuperscript{Gal4} alleles was re-inserted (Fig. \ref{fig:3}A).
DmCa\textsubscript{v}3\textsuperscript{Gal4} homozygote flies were viable and fertile with normal appearance and no movement defect.
In mammals, two T-type channel subtypes of a1G and a1I involve in generation of neural oscillation in NREM sleep.
It is well established that flies have sleep-like state and share conserved mechanisms with mammals although it is still not clear whether flies have distinct sleep stages.

DmCa\textsubscript{v}3\textsuperscript{Gal4} flies show increased sleep that is particularly prominent in subjective daytime under continuous dark (DD) conditions (Fig. \ref{fig:3}B).
Subjective daytime sleep in DmCa\textsubscript{v}3\textsuperscript{Rescue} flies, however, is restored to a level similar to that of w\textsuperscript{1118} control flies (Fig. \ref{fig:3}C).
By measuring waking locomotor activity, we were able to confirm that the increased sleep of DmCa\textsubscript{v}3\textsuperscript{Gal4} flies is not an artifact of a generalized reduction in movement. In fact, DmCa\textsubscript{v}3\textsuperscript{Gal4} show slightly higher levels of waking activity than their respective controls (Fig. \ref{fig:3}D).
Fly sleep consists of a number of sleep episodes.
We examined whether increased sleep amount is due to prolonged duration or increased number of episodes.
DmCa\textsubscript{v}3\textsuperscript{Gal4} showed either increased episode number or length in light phase in LD and subjective light phase in DD (Fig. \ref{fig:4}F and G).
On the other hand, sleep is consolidated in dark phase in LD and subjective dark phase in DD as bout number was deceased and bout length was increased in DmCa\textsubscript{v}3\textsuperscript{Gal4}, which were restored in DmCa\textsubscript{v}3\textsuperscript{Rescue} (Fig. \ref{fig:4}I and J).
These results suggest DmCa\textsubscript{v}3\textsuperscript{Gal4} mutant have significantly higher sleep need compared to control.
We also generated independent deletion mutants by P-element based imprecise excision to exclude the possibility that ectopic expression of Gal4 in DmCa\textsubscript{v}3\textsuperscript{Gal4} might cause sleep phenotype.
All the deletion mutants and their trans-hetero mutants showed increased sleep in DD (Fig. \ref{fig:S3}), supporting DmCa\textsubscript{v}3 channel regulates sleep state.