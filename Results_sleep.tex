\subsection*{DmCa\textsubscript{v}3 mutants show increased sleep}

Since the Gal4 coding sequence inserted into \emph{DmCa\textsubscript{v}3\textsuperscript{Founder}} to produce the \emph{DmCa\textsubscript{v}3\textsuperscript{Gal4}} allele included a termination sequence (Fig. \ref{fig:3}a), DmCa\textsubscript{v}3\textsuperscript{Gal4} is likely a null allele.
As expected, we were unable to detect DmCa\textsubscript{v}3 expression in the fly head lysates \emph{DmCa\textsubscript{v}3\textsuperscript{Gal4}} flies in western blot analyses using polyclonal DmCa\textsubscript{v}3-specific antisera (Fig. \ref{fig:3}a). We did, however, detect strong DmCa\textsubscript{v}3 expression in fly lysates from \emph{w\textsuperscript{1118}} controls and from a \emph{DmCa\textsubscript{v}3\textsuperscript{Rescue}} allele in which the fragment deleted in both the \emph{DmCa\textsubscript{v}3\textsuperscript{Founder}} and \emph{DmCa\textsubscript{v}3\textsuperscript{Gal4}} alleles was re-inserted (Fig. \ref{fig:3}a and b).
\emph{DmCa\textsubscript{v}3\textsuperscript{Gal4}} homozygote flies were viable and fertile with normal appearance and no movement defect.
In mammals, two T-type channel subtypes of a1G and a1I involve in generation of neural oscillation in NREM sleep.
It is well established that flies have sleep-like state and share conserved mechanisms with mammals although it is still not clear whether flies have distinct sleep stages.

\emph{DmCa\textsubscript{v}3\textsuperscript{Gal4}} flies show increased sleep that is particularly prominent in subjective daytime under continuous dark (DD) conditions (Fig. \ref{fig:3}c and d).
Increased sleep in \emph{DmCa\textsubscript{v}3\textsuperscript{Rescue}} flies, however, is restored to a level similar to that of \emph{w\textsuperscript{1118}} control flies (Fig. \ref{fig:3}c and d).
By measuring waking locomotor activity, we were able to confirm that the increased sleep of \emph{DmCa\textsubscript{v}3\textsuperscript{Gal4}} flies is not an artifact of a generalized reduction in movement. In fact, \emph{DmCa\textsubscript{v}3\textsuperscript{Gal4}} show slightly higher levels of waking activity than their respective controls (Fig. \ref{fig:3}e).
Fly sleep consists of a number of sleep episodes.
We examined whether increased sleep amount is due to prolonged duration or increased number of episodes.
The number of sleep bout in \emph{DmCa\textsubscript{v}3\textsuperscript{Gal4}} was decreased only in LD which was not restored in \emph{DmCa\textsubscript{v}3\textsuperscript{Rescue}} (Fig. \ref{fig:3}f).
On the other hand, a averaged sleep bout length was increased both in LD and DD which was restored in DmCa\textsubscript{v}3\textsuperscript{Rescue} (Fig. \ref{fig:3}g).
These results suggest \emph{DmCa\textsubscript{v}3\textsuperscript{Gal4}} mutant have significantly higher sleep need compared to control.
We also generated independent deletion mutants by P-element based imprecise excision to exclude the possibility that ectopic expression of Gal4 in \emph{DmCa\textsubscript{v}3\textsuperscript{Gal4}} might cause sleep phenotype.
All the deletion mutants and their trans-hetero mutants showed increased sleep in DD (Fig. \ref{fig:S2}), supporting DmCa\textsubscript{v}3 channel regulates sleep state.
    
  
  