\section*{Results}

\subsection*{DmCa\textsubscript{v}3 channel produces LVA currents in \emph{Xenopus} oocytes}

We cloned a full length fly T-type Ca\textsuperscript{2+} channel cDNA (named DmCa\textsubscript{v}3) from a fly head cDNA library.
The sequence of cloned cDNA was exactly matched to an isoform predicted in FlyBase (\href{http://}{http://flybase.org}). 

To examine the electrophysiological properties of DmCa\textsubscript{v}3, we injected cRNAs made from the DmCa\textsubscript{v}3 cDNA template into \emph{Xenopus} oocytes.
From 4 days after cRNA injection, DmCa\textsubscript{v}3 was expressed as measured by robust inward currents in 10 mM Ba\textsuperscript{2+} as a charge carrier.
To directly compare the biophysical properties of DmCa\textsubscript{v}3 and mammalian T-type Ca\textsuperscript{2+} channel homolog under the same conditions, we expressed rat Ca\textsubscript{v}3.1 subunit of which biophysical properties were previously reported in the expression system\cite{9495342}.
Compared to Ca\textsubscript{v}3.1, DmCa\textsubscript{v}3 had a similar low-voltage threshold (around \textminus60 mV) for activation, although the averaged value is slightly lower by 3\sim 4 mV.
The current traces of DmCa\textsubscript{v}3 and Ca\textsubscript{v}3.1 were activated and then inactivated during serial step pulses from a holding potential of \textminus90 mV, producing transient current kinetics with the inactivation kinetics of DmCa\textsubscript{v}3 currents being likely to be slightly slower than those of Ca\textsubscript{v}3.1 currents.
The activation and inactivation kinetics of currents through DmCa\textsubscript{v}3 was accelerated as the higher step pulses were applied, producing a criss-crossing pattern (Fig. \ref{fig:1}a), a typical T-type Ca\textsuperscript{2+} channel kinetics.
Analysis of current-voltage (I-V) relationships showed that V\textsubscript{50,act} for half-maximal activation and slope factor (k) of DmCa\textsubscript{v}3 channel are \textminus43.32 $\pm$ 1.38 mV and 7.74 $\pm$ 1.33, while those of Ca\textsubscript{v}3.1 are \textminus38.92 $\pm$ 0.99 and 6.35 $\pm$ 0.94 (Fig. \ref{fig:1}a).
These results suggest that DmCa\textsubscript{v}3 can be activated more negative potential than Ca\textsubscript{v}3.1 by 4.4 mV.
Taken together, the biophysical properties of DmCa\textsubscript{v}3 including activation threshold of about \textminus60 mV, formation of maximal current amplitude at \textminus20 mV, transient current kinetics, a criss-crossing pattern by currents evoked by a voltage protocol for I-V are very similar to the hallmark properties of native T-type Ca\textsuperscript{2+} channels as well as cloned channels\cite{9495342, 6087159, 9670923, 10066244}.

The activation curves obtained from fitting chord conductance with a Boltzmann equation showed that the potentials (V\textsubscript{50,act}) for half-maximal activation of DmCa\textsubscript{v}3 and Ca\textsubscript{v}3.1 are \textminus43.32 $\pm$ 1.58 and \textminus38.92 $\pm$ 1.15 mV, respectively, indicating that DmCa\textsubscript{v}3 activated at 4.4 mV lower test potentials than Ca\textsubscript{v}3.1 (P $<$ 0.05, Student's t-test, n=10 -- 14) (Fig. \ref{fig:1}b and Table \ref{tab:1}).
In steady-state inactivation, the potentials (V\textsubscript{50,inact}) of 50\% channel availability for DmCa\textsubscript{v}3 and Ca\textsubscript{v}3.1 are estimated to be \textminus58.04 $\pm$ 0.71 and \textminus61.31 $\pm$ 0.70 mV (P $<$ 0.05, Student's t-test, n=5 -- 15), indicating that the V\textsubscript{50,inact} of DmCa\textsubscript{v}3 is 3.3 mV more positive than that of Cav\textsubscript{v}3.1 (Fig. \ref{fig:1}b and Table \ref{tab:1}).
In regard to window current typically designated by the portion overlapped in the steady-state inactivation and activation curves, the window region for DmCa\textsubscript{v}3 is significantly larger than that for Ca\textsubscript{v}3.1, implying that DmCa\textsubscript{v}3 is capable of persistently evoking higher channel activity over relevant voltage range than Ca\textsubscript{v}3.1.

Voltage-dependent kinetics are known to different among three mammalian T-type calcium channels\cite{10594642}.
Ca\textsubscript{v}3.1 and Ca\textsubscript{v}3.2 shows similar activation/inactivation kinetics whereas Ca\textsubscript{v}3.3 has much slower kinetics.
We compared the time constants of activation and inactivation by fitting the current traces with double exponential function.
At the test potentials from \textminus50 mV to +20 mV,  DmCa\textsubscript{v}3 has slower current kinetics than Ca\textsubscript{v}3.1 (P $<$ 0.01 or 0.001, Fig. \ref{fig:1}c).
For example, the activation and inactivation time constants of DmCa\textsubscript{v}3 current at \textminus20 mV test potential are 2.2 $\pm$ 0.2 ms and 23.4 $\pm$ 1.4 ms respectively, showing about 2-fold slower activation and inactivation kinetics than those of rat Ca\textsubscript{v}3.1 current (Table \ref{tab:1}).

One of defining properties of T-type calcium channels is that they deactivate slowly compared to the HVA calcium channels that have much faster deactivation kinetics\cite{9495342, 10066244, 2419479}. 
For characterizing the deactivation kinetics of DmCa\textsubscript{v}3, we transiently expressed DmCa\textsubscript{v}3 cDNA in HEK-293 cells and recorded tail currents using whole cell patch clamping. 
The tail currents of DmCa\textsubscript{v}3 appeared to be slowly decayed in a voltage dependent manner (Fig. \ref{fig:1}d). 
The deactivation time constants from fitting the tail currents are most similar to mammalian Ca\textsubscript{v}3.3 among the three Ca\textsubscript{v}3 isoforms previously reported (Table \ref{tab:1}). 

Previous studies have shown that the current amplitude of Ca\textsubscript{v}3.1 in Ca\textsuperscript{2+} as a charge carrier is greater than in equi-molar Ba\textsuperscript{2+}, while the opposite is true for Ca\textsubscript{v}3.2 or Ca\textsubscript{v}3.3\cite{mcrory:2000aa,shcheglovitov:2007aa}.
Herein, we determined relative permeability (I\textsubscript{Ca}/I\textsubscript{Ba}) of Ca\textsuperscript{2+} and Ba\textsuperscript{2+} ions through DmCa\textsubscript{v}3 or Ca\textsubscript{v}3.1.
As reported, the I\textsubscript{Ca}/I\textsubscript{Ba} ratio of Ca\textsubscript{v}3.1 is 1.55 $\pm$ 0.03 (n=3), showing that the peak current amplitude of Ca\textsubscript{v}3.1 is greater in 10 mM Ca\textsuperscript{2+} than 10 mM Ba\textsuperscript{2+} (Fig. \ref{fig:1}e).
On the contrary, the I\textsubscript{Ca}/I\textsubscript{Ba} of DmCa\textsubscript{v}3 channel is 0.68 $\pm$ 0.04, indicating that the current amplitude through DmCa\textsubscript{v}3 channels in 10 mM Ca\textsuperscript{2+} solution is smaller than in 10 mM Ba\textsuperscript{2+} solution (n=6) (Fig. \ref{fig:1}e).
These findings suggest that the relative permeability (I\textsubscript{Ca}/I\textsubscript{Ba}) of DmCa\textsubscript{v}3 channel resembles the property of Ca\textsubscript{v}3.2 or Ca\textsubscript{v}3.3 rather than Ca\textsubscript{v}3.1 (Table \ref{tab:1}).
We also compared relative permeability of the two divalent ions through DmCa\textsubscript{v}3 and Ca\textsubscript{v}3.1 using their maximal slope conductance ratios (G\textsubscript{MaxCa}/G\textsubscript{MaxBa}).
The G\textsubscript{MaxCa}/G\textsubscript{MaxBa} ratios for DmCa\textsubscript{v}3 and Ca\textsubscript{v}3.1 are 0.71 $\pm$ 0.10 and 1.43 $\pm$ 0.17 respectively (n=6, 4), supporting that Ca\textsuperscript{2+} permeation through DmCa\textsubscript{v}3 channel are less than Ba\textsuperscript{2+} permeation (Fig. \ref{fig:1}e).

Sensitivity to Ni\textsuperscript{2+}, a blocker of T-type Ca\textsuperscript{2+} channel, has been a criteria that is differential according to T-type channel isoforms\cite{10585925}.
Among three subtypes, Ca\textsubscript{v}3.2 is more sensitive to Ni\textsuperscript{2+} than Ca\textsubscript{v}3.1 and Ca\textsubscript{v}3.3.
Application of low micro-molar concentrations of Ni\textsuperscript{2+} solutions inhibited DmCa\textsubscript{v}3 currents in a concentration dependent manner (Fig. \ref{fig:1}f).
In contrast, Ca\textsubscript{v}3.1 currents required much higher concentrations of Ni\textsuperscript{2+} solutions to be blocked (Fig. \ref{fig:1}f).
Analysis of dose-response curves shows that the IC\textsubscript{50} values for DmCa\textsubscript{v}3 and Ca\textsubscript{v}3.1 are 5.12 and 276.5 $\mu$M, respectively.
These results indicate that the Ni\textsuperscript{2+} block of DmCa\textsubscript{v}3 currents is ~50 fold more sensitive than that of Ca\textsubscript{v}3.1 , suggesting that the nickel sensitive block of DmCa\textsubscript{v}3 channel resembles Ca\textsubscript{v}3.2 T-type channel (Table \ref{tab:1}).