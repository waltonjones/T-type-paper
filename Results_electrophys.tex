\section*{Results}

\subsection*{DmCa\textsubscript{v}3 channel produces LVA currents in \emph{Xenopus} oocytes}

The fly T-type Ca\textsuperscript{2+} channel DmCa\textsubscript{v}3 spans roughly 90 kilobases of genomic DNA and has five different annotated mRNA transcripts designated RB through RF.
Because the smallest of these transcripts is still over 10 kilobases in length, we used a piece-meal approach to assemble a full-length cDNA.
To do so, we isolated total RNA from fly heads and used reverse transcription to produce cDNAs.
Using these cDNAs as a template, we amplified and then assembled partial clones to obtain full-length cDNAs for both the RB (NM_132068) and RC (NM_001103419) DmCa\textsubscript{v}3 transcripts.
After sequence verification, we used these DmCa\textsubscript{v}3 cDNA clones to produce cRNAs for injection into \emph{Xenopus} oocytes.
We were able to confirm expression of the RC isoform, but not the RB isoform, by measuring robust inward currents using 10 mM Ba\textsuperscript{2+} as a charge carrier 4 days after cRNA injection. 
In all subsequent experiments performed with this RC isoform cDNA, we refer to it simply as DmCa\textsubscript{v}3.

We next compared the biophysical properties of DmCa\textsubscript{v}3 with those of a mammalian T-type Ca\textsuperscript{2+} channel homolog, rat Ca\textsubscript{v}3.1\cite{PerezReyes:1998gn}, using the \emph{Xenopus} oocyte expression system.
Both DmCa\textsubscript{v}3 and Ca\textsubscript{v}3.1 have low-voltage activation thresholds, but the threshold of DmCa\textsubscript{v}3 (\textminus60 mV) is slighty lower than that of the rat channel by 3 \sim 4 mV.
Both DmCa\textsubscript{v}3 and Ca\textsubscript{v}3.1 produce current kinetics typical of T-type Ca\textsuperscript{2+} channels when subjected to a protocol of serial step pulses from a holding potential of \textminus90 mV.
Although the inactivation kinetics of DmCa\textsubscript{v}3 are slightly slower than those of Ca\textsubscript{v}3.1, both the activation and inactivation kinetics of currents through DmCa\textsubscript{v}3 accelerate at higher step pulse values. This produces the criss-crossing pattern typical of T-type Ca\textsuperscript{2+} channels (Fig. \ref{fig:1}a).
Together, these biophysical properties---an activation threshold of \textminus60 mV, a potential of maximal current at \textminus20 mV, transient current kinetics, a criss-crossing pattern in currents evoked by a step pulse voltage protocol---all of these properties mark DmCa\textsubscript{v}3 as a typical T-type Ca\textsuperscript{2+} channel\cite{PerezReyes:1998gn,carbone:1984aa,Cribbs:1998vc,lee:1999aa}.

We next obtained activation curves for DmCa\textsubscript{v}3 and Ca\textsubscript{v}3.1 by fitting chord conductances with the Boltzmann equation. The potential for half-maximal activation (V\textsubscript{50,act}) of DmCa\textsubscript{v}3 and Ca\textsubscript{v}3.1 are \textminus43.32 $\pm$ 1.58 and \textminus38.92 $\pm$ 1.15 mV, respectively. This indicates that DmCa\textsubscript{v}3 is activated at 4.4 mV lower test potentials than Ca\textsubscript{v}3.1 (\emph{p}$<$ 0.05, Student's t-test, n=11 -- 14) (Fig. \ref{fig:1}b and Table \ref{tab:1}).
During steady-state inactivation, the potentials of 50\% channel availability (V\textsubscript{50,inact}) for DmCa\textsubscript{v}3 and Ca\textsubscript{v}3.1 are estimated to be \textminus58.04 $\pm$ 0.71 and \textminus61.31 $\pm$ 0.70 mV (\emph{p}$<$ 0.05, Student's t-test, n=5 -- 15). In other words, the V\textsubscript{50,inact} of DmCa\textsubscript{v}3 is 3.3 mV more positive than that of Cav\textsubscript{v}3.1 (Fig. \ref{fig:1}b and Table \ref{tab:1}).
An ion channel's so-called ``window current'' is the range of overlap in its steady-state activation and inactivation curves. This window for DmCa\textsubscript{v}3 is significantly larger than that of Ca\textsubscript{v}3.1, implying that DmCa\textsubscript{v}3 is capable of persistently passing larger currents over the relevant voltage range than Ca\textsubscript{v}3.1.

The voltage-dependent kinetics of the three mammalian T-type calcium channels are known to differ, with Ca\textsubscript{v}3.1 and Ca\textsubscript{v}3.2 showing faster activation/inactivation kinetics than Ca\textsubscript{v}3.3\cite{klockner:1999aa}.
To compare the time constants of activation and inactivation for DmCa\textsubscript{v}3 and Ca\textsubscript{v}3.1, we fitted current traces from each with a double exponential function.
At test potentials ranging from \textminus50 mV to +20 mV,  DmCa\textsubscript{v}3 has slower current kinetics than Ca\textsubscript{v}3.1 (\emph{p}$<$ 0.01 or 0.001, Fig. \ref{fig:1}c).
For example, the activation and inactivation time constants of DmCa\textsubscript{v}3 current at a \textminus20 mV test potential are 2.2 $\pm$ 0.2 ms and 23.4 $\pm$ 1.4 ms respectively. This means the activation and inactivation kinetics of DmCa\textsubscript{v}3 are 2-fold slower than those of rat Ca\textsubscript{v}3.1, but still in the ``fast'' range (Table \ref{tab:1}).

Another defining property of the LVA T-type calcium channels is that they deactivate much more slowly than HVA calcium channels\cite{PerezReyes:1998gn,lee:1999aa,matteson:1986aa}. 
To characterize the deactivation kinetics of DmCa\textsubscript{v}3, we performed a transient transfection of the DmCa\textsubscript{v}3 cDNA into HEK-293 cells followed by whole-cell patch clamp recordings of tail currents.
As expected, the tail currents of DmCa\textsubscript{v}3 appear to undergo a slow voltage-dependent decay (Fig. \ref{fig:1}d).
The deactivation time constant obtained for DmCa\textsubscript{v}3 (0.93 $\pm$ 0.14 ms) by curve-fitting the tail currents is in the same range as that reported for all three mammalian Ca\textsubscript{v}3 isoforms (Table \ref{tab:1}). 

Previous studies have shown that Ca\textsubscript{v}3.1 passes larger amplitude currents when Ca\textsuperscript{2+} is used as a charge carrier rather than equimolar Ba\textsuperscript{2+}, but that the opposite is true for Ca\textsubscript{v}3.2 and Ca\textsubscript{v}3.3\cite{mcrory:2000aa,shcheglovitov:2007aa}.
We, thus, measured the relative permeability of DmCa\textsubscript{v}3 and Ca\textsubscript{v}3.1 to Ca\textsuperscript{2+} and Ba\textsuperscript{2+} ions (I\textsubscript{Ca}/I\textsubscript{Ba}).
As previously reported, we measured a I\textsubscript{Ca}/I\textsubscript{Ba} ratio for Ca\textsubscript{v}3.1 of 1.55 $\pm$ 0.03 (n=3). This means the peak current amplitude of Ca\textsubscript{v}3.1 is greater in 10 mM Ca\textsuperscript{2+} than 10 mM Ba\textsuperscript{2+} (Fig. \ref{fig:1}e).
The I\textsubscript{Ca}/I\textsubscript{Ba} ratio for DmCa\textsubscript{v}3, however, is 0.68 $\pm$ 0.04. This means DmCa\textsubscript{v}3 passes a smaller current in 10 mM Ca\textsuperscript{2+} than in 10 mM Ba\textsuperscript{2+} (n=6) (Fig. \ref{fig:1}e).
We also compared the relative permeability of DmCa\textsubscript{v}3 and Ca\textsubscript{v}3.1 to these two divalent cations by comparing their maximal slope conductance ratios (G\textsubscript{MaxCa}/G\textsubscript{MaxBa}).
The G\textsubscript{MaxCa}/G\textsubscript{MaxBa} ratios for DmCa\textsubscript{v}3 and Ca\textsubscript{v}3.1 are 0.71 $\pm$ 0.10 (n=6) and 1.43 $\pm$ 0.17 (n=4), respectively (Fig. \ref{fig:1}e).
Thus, in terms of its relative permeability to Ca\textsuperscript{2+} and Ba\textsuperscript{2+}, DmCa\textsubscript{v}3 is more similar to Ca\textsubscript{v}3.2 or Ca\textsubscript{v}3.3 than Ca\textsubscript{v}3.1 (Table \ref{tab:1}).

Finally, T-type channel isoforms are also known to be differentially sensitive to blockage by Ni\textsuperscript{2+} ions, with Ca\textsubscript{v}3.2 being the most sensitive of the three Ca\textsubscript{v}3 isoforms\cite{lee:1999ab}.
Low micromolar levels of Ni\textsuperscript{2+} produce a concentration-dependent inhibition of DmCa\textsubscript{v}3 (IC\textsubscript{50} = 5.12 $\mu$M), while much higher levels of Ni\textsuperscript{2+} are necessary for significant blockage of Ca\textsubscript{v}3.1 (IC\textsubscript{50} = 276.5 $\mu$M) (Fig. \ref{fig:1}f).
Thus, in terms of Ni\textsuperscript{2+} sensitivity, DmCa\textsubscript{v}3 more closely resembles Ca\textsubscript{v}3.2 than Ca\textsubscript{v}3.1 (Table \ref{tab:1}).
