\section*{Results}

\subsection*{DmCa\textsubscript{v}3 channel produces LVA currents in \emph{Xenopus} oocytes}

The fly T-type Ca\textsuperscript{2+} channel DmCa\textsubscript{v}3 spans roughly 90 kilobases of genomic DNA and has five different annotated mRNA transcripts designated RB through RF.
Because the smallest of these transcripts is still over 10 kilobases in length, we used a piece-meal approach to assemble a full-length cDNA.
To do so, we isolated total RNA from fly heads and used reverse transcription to produce cDNAs.
Using these cDNAs as a template, we amplified and then assembled partial clones to obtain full-length cDNAs for both the RB (NM_132068) and RC (NM_001103419) DmCa\textsubscript{v}3 transcripts.
After sequence verification, we used these DmCa\textsubscript{v}3 cDNA clones to produce cRNAs for injection into \emph{Xenopus} oocytes.
We were able to confirm expression of the RC isoform, but not the RB isoform, by measuring robust inward currents using 10 mM Ba\textsuperscript{2+} as a charge carrier 4 days after cRNA injection. 
In all subsequent experiments performed with this RC isoform cDNA, we refer to it simply as DmCa\textsubscript{v}3.

We next compared the biophysical properties of DmCa\textsubscript{v}3 with those of a mammalian T-type Ca\textsuperscript{2+} channel homolog, rat Ca\textsubscript{v}3.1\cite{PerezReyes:1998gn}, using the \emph{Xenopus} oocyte expression system.
Both DmCa\textsubscript{v}3 and Ca\textsubscript{v}3.1 have low-voltage activation thresholds, but the threshold of DmCa\textsubscript{v}3 (\textminus60 mV) is slighty lower than that of the rat channel by 3 \sim 4 mV.
Both DmCa\textsubscript{v}3 and Ca\textsubscript{v}3.1 produce current kinetics typical of T-type Ca\textsuperscript{2+} channels when subjected to a protocol of serial step pulses from a holding potential of \textminus90 mV.
Although the inactivation kinetics of DmCa\textsubscript{v}3 are slightly slower than those of Ca\textsubscript{v}3.1, both the activation and inactivation kinetics of currents through DmCa\textsubscript{v}3 accelerate at higher step pulse values. This produces the criss-crossing pattern typical of T-type Ca\textsuperscript{2+} channels (Fig. \ref{fig:1}a).
According to a comparison of the current-voltage (I-V) curves, the V\textsubscript{50,act} for half-maximal activation and the slope factor (k) of DmCa\textsubscript{v}3 are \textminus43.32 $\pm$ 1.38 mV and 7.74 $\pm$ 1.33, while those of Ca\textsubscript{v}3.1 are \textminus38.92 $\pm$ 0.99 and 6.35 $\pm$ 0.94 (Fig. \ref{fig:1}a).
This suggests DmCa\textsubscript{v}3 is activated at slightly  more negative potentials (a difference of 4.4 mV) than Ca\textsubscript{v}3.1.
Together, these biophysical properties---an activation threshold of \textminus60 mV, a potential of maximal current at \textminus20 mV, transient current kinetics, a criss-crossing pattern in currents evoked by a step pulse voltage protocol---all of these properties mark DmCa\textsubscript{v}3 as a typical T-type Ca\textsuperscript{2+} channel\cite{PerezReyes:1998gn,carbone:1984aa,Cribbs:1998vc,lee:1999aa}.

We obtained activation curves for DmCa\textsubscript{v}3 and Ca\textsubscript{v}3.1 by fitting chord conductances with the Boltzmann equation. The potential for half-maximal activation (V\textsubscript{50,act}) of DmCa\textsubscript{v}3 and Ca\textsubscript{v}3.1 are \textminus43.32 $\pm$ 1.58 and \textminus38.92 $\pm$ 1.15 mV, respectively. This indicates that DmCa\textsubscript{v}3 is activated at 4.4 mV lower test potentials than Ca\textsubscript{v}3.1 (P $<$ 0.05, Student's t-test, n=10 -- 14) (Fig. \ref{fig:1}b and Table \ref{tab:1}).
During steady-state inactivation, the potentials of 50\% channel availability (V\textsubscript{50,inact}) for DmCa\textsubscript{v}3 and Ca\textsubscript{v}3.1 are estimated to be \textminus58.04 $\pm$ 0.71 and \textminus61.31 $\pm$ 0.70 mV (P $<$ 0.05, Student's t-test, n=5 -- 15). In other words, the V\textsubscript{50,inact} of DmCa\textsubscript{v}3 is 3.3 mV more positive than that of Cav\textsubscript{v}3.1 (Fig. \ref{fig:1}b and Table \ref{tab:1}).
An ion channel's so-called ``window current'' is the range of overlap in its steady-state activation and inactivation curves. This window for DmCa\textsubscript{v}3 is significantly larger than that of Ca\textsubscript{v}3.1, implying that DmCa\textsubscript{v}3 is capable of persistently passing larger currents over the relevant voltage range than Ca\textsubscript{v}3.1.

Voltage-dependent kinetics are known to different among three mammalian T-type calcium channels\cite{klockner:1999aa}.
Ca\textsubscript{v}3.1 and Ca\textsubscript{v}3.2 shows similar activation/inactivation kinetics whereas Ca\textsubscript{v}3.3 has much slower kinetics.
We compared the time constants of activation and inactivation by fitting the current traces with double exponential function.
At the test potentials from \textminus50 mV to +20 mV,  DmCa\textsubscript{v}3 has slower current kinetics than Ca\textsubscript{v}3.1 (P $<$ 0.01 or 0.001, Fig. \ref{fig:1}c).
For example, the activation and inactivation time constants of DmCa\textsubscript{v}3 current at \textminus20 mV test potential are 2.2 $\pm$ 0.2 ms and 23.4 $\pm$ 1.4 ms respectively, showing about 2-fold slower activation and inactivation kinetics than those of rat Ca\textsubscript{v}3.1 current (Table \ref{tab:1}).

One of defining properties of T-type calcium channels is that they deactivate slowly compared to the HVA calcium channels that have much faster deactivation kinetics\cite{PerezReyes:1998gn,lee:1999aa,matteson:1986aa}. 
For characterizing the deactivation kinetics of DmCa\textsubscript{v}3, we transiently expressed DmCa\textsubscript{v}3 cDNA in HEK-293 cells and recorded tail currents using whole cell patch clamping. 
The tail currents of DmCa\textsubscript{v}3 appeared to be slowly decayed in a voltage dependent manner (Fig. \ref{fig:1}d). 
The deactivation time constants from fitting the tail currents are most similar to mammalian Ca\textsubscript{v}3.3 among the three Ca\textsubscript{v}3 isoforms previously reported (Table \ref{tab:1}). 

Previous studies have shown that the current amplitude of Ca\textsubscript{v}3.1 in Ca\textsuperscript{2+} as a charge carrier is greater than in equi-molar Ba\textsuperscript{2+}, while the opposite is true for Ca\textsubscript{v}3.2 or Ca\textsubscript{v}3.3\cite{mcrory:2000aa,shcheglovitov:2007aa}.
Herein, we determined relative permeability (I\textsubscript{Ca}/I\textsubscript{Ba}) of Ca\textsuperscript{2+} and Ba\textsuperscript{2+} ions through DmCa\textsubscript{v}3 or Ca\textsubscript{v}3.1.
As reported, the I\textsubscript{Ca}/I\textsubscript{Ba} ratio of Ca\textsubscript{v}3.1 is 1.55 $\pm$ 0.03 (n=3), showing that the peak current amplitude of Ca\textsubscript{v}3.1 is greater in 10 mM Ca\textsuperscript{2+} than 10 mM Ba\textsuperscript{2+} (Fig. \ref{fig:1}e).
On the contrary, the I\textsubscript{Ca}/I\textsubscript{Ba} of DmCa\textsubscript{v}3 channel is 0.68 $\pm$ 0.04, indicating that the current amplitude through DmCa\textsubscript{v}3 channels in 10 mM Ca\textsuperscript{2+} solution is smaller than in 10 mM Ba\textsuperscript{2+} solution (n=6) (Fig. \ref{fig:1}e).
These findings suggest that the relative permeability (I\textsubscript{Ca}/I\textsubscript{Ba}) of DmCa\textsubscript{v}3 channel resembles the property of Ca\textsubscript{v}3.2 or Ca\textsubscript{v}3.3 rather than Ca\textsubscript{v}3.1 (Table \ref{tab:1}).
We also compared relative permeability of the two divalent ions through DmCa\textsubscript{v}3 and Ca\textsubscript{v}3.1 using their maximal slope conductance ratios (G\textsubscript{MaxCa}/G\textsubscript{MaxBa}).
The G\textsubscript{MaxCa}/G\textsubscript{MaxBa} ratios for DmCa\textsubscript{v}3 and Ca\textsubscript{v}3.1 are 0.71 $\pm$ 0.10 and 1.43 $\pm$ 0.17 respectively (n=6, 4), supporting that Ca\textsuperscript{2+} permeation through DmCa\textsubscript{v}3 channel are less than Ba\textsuperscript{2+} permeation (Fig. \ref{fig:1}e).

Sensitivity to Ni\textsuperscript{2+}, a blocker of T-type Ca\textsuperscript{2+} channel, has been a criteria that is differential according to T-type channel isoforms\cite{lee:1999ab}.
Among three subtypes, Ca\textsubscript{v}3.2 is more sensitive to Ni\textsuperscript{2+} than Ca\textsubscript{v}3.1 and Ca\textsubscript{v}3.3.
Application of low micro-molar concentrations of Ni\textsuperscript{2+} solutions inhibited DmCa\textsubscript{v}3 currents in a concentration dependent manner (Fig. \ref{fig:1}f).
In contrast, Ca\textsubscript{v}3.1 currents required much higher concentrations of Ni\textsuperscript{2+} solutions to be blocked (Fig. \ref{fig:1}f).
Analysis of dose-response curves shows that the IC\textsubscript{50} values for DmCa\textsubscript{v}3 and Ca\textsubscript{v}3.1 are 5.12 and 276.5 $\mu$M, respectively.
These results indicate that the Ni\textsuperscript{2+} block of DmCa\textsubscript{v}3 currents is ~50 fold more sensitive than that of Ca\textsubscript{v}3.1 , suggesting that the nickel sensitive block of DmCa\textsubscript{v}3 channel resembles Ca\textsubscript{v}3.2 T-type channel (Table \ref{tab:1}).
  
  
  