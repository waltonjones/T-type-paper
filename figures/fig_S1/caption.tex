\label{fig:S1}
\textbf{Generation of DmCa\textsubscript{v}3 targeting alleles.}
\\
\textbf{A.} Schematic representation of DmCa\textsubscript{v}3 gene locus.
Coding exons are shown in red and start codon in 4th exon is shown.
\textbf{B.} Gene targeting by homologous recombination.
About 2kb of genomic region including the 4th exon with flanking introns was replaced by targeting construct containing attP sequence for site-specific recombination and loxP-flanked white+ marker.
White+ marker was removed by Cre-recombinase for further modification.
\textbf{C.} Schematics of DmCa\textsubscript{v}3\textsuperscript{Gal4}, DmCa\textsubscript{v}3\textsuperscript{Rescue} and GFP::DmCa\textsubscript{v}3 alleles.
Modified sequences for each alleles were inserted in integration vector (pGE-attB) carrying attB site, then integrated into attP site of DmCa\textsubscript{v}3\textsuperscript{Founder} line through $\phi$C31-mediated DNA integration.
DmCa\textsubscript{v}3\textsuperscript{Gal4} has Gal4 coding sequence flanked by splicing acceptor (SA) in 5' upstream and polyA termination sequence (pA) in 3' downstream.
In DmCa\textsubscript{v}3\textsuperscript{Rescue}, deleted exons and flanking introns of DmCa\textsubscript{v}3\textsuperscript{Founder} were reintroduced.
GFP:: DmCa\textsubscript{v}3 allele was generated by inserting multi-tags (EGFP-FlAsH-StrepII-3xFlag) and following linker sequence (GlyGlySer)4 after the ATG start codon of 4th exon in DmCa\textsubscript{v}3\textsuperscript{Rescue} construct in pGE-attB and integrating the vector into founder line.
Vector sequences including white marker was removed by Cre recombinase, leaving one loxP site.