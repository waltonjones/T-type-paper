\label{fig:1}
\textbf{Comparing the biophysical properties of Ca-\alpha1T and rat Ca\textsubscript{v}3.1.}
\\
\textbf{(a)} (Left) Representative current traces through Ca-\alpha1T and Ca\textsubscript{v}3.1 expressed in \emph{Xenopus} oocytes. 
In 10 mM Ba\textsuperscript{2+}, currents were elicited by depolarizing step pulses separated by 10 mV increments from $-$70 mV to +40 mV from a holding potential of $-$90 mV. 
(Right) I-V relationships of Ca-\alpha1T and Ca\textsubscript{v}3.1. 
Peak currents for each oocyte were normalized to the maximum current. 
Percent amplitudes from oocytes expressing Ca-\alpha1T ($\fullmoon$) or Ca\textsubscript{v}3.1 ($\square$) plotted against the test potentials and fitted with the Boltzmann equation. 
\textbf{(b)} (Left) Steady-state inactivation was measured during voltage steps to $-$20 mV after 10 s prepulses to potentials between $-$100 mV and $-$40 mV. 
(Right) Voltage-dependent activation and steady-state inactivation curves of Ca-\alpha1T ($\fullmoon$, $\newmoon$) and Ca\textsubscript{v}3.1 ($\square$, $\blacksquare$) plotted and fitted with the Boltzmann equation. 
\textbf{(c)} The activation time constant ($\tau$\textsubscript{act}) and inactivation time constant ($\tau$\textsubscript{inact}) of Ca-\alpha1T ($\fullmoon$) and Ca\textsubscript{v}3.1 ($\square$) were obtained by curve fitting the current traces with double exponentials. 
\textbf{(d)} Voltage-dependent deactivation of Ca-\alpha1T in HEK-293 cells.
Tail currents were elicited by application of repeated step pulses to $-$20 mV for 10 ms, followed by various re-polarizing potentials ranging from $-$120 mV to $-$50 mV. 
Deactivation time constants were obtained by fitting tail current traces with a single exponential and plotted against re-polarizing potentials. 
\textbf{(e)} I\textsubscript{Ca}/I\textsubscript{Ba} ratios of Ca-\alpha1T and Ca\textsubscript{v}3.1. 
(Left) Representative current traces through Ca-\alpha1T and Ca\textsubscript{v}3.1 measured in 10 mM Ba\textsuperscript{2+} or 10 mM Ca\textsuperscript{2+} elicited by 10 mV step pulses from a holding potential of $-$90 mV.
Ba\textsuperscript{2+} currents are in black; Ca\textsuperscript{2+} currents are in grey. 
(Middle) I-V relationships of Ca-\alpha1T ($\fullmoon$}, $\newmoon$) and Ca\textsubscript{v}3.1 ($\square$, $\blacksquare$) in 10 mM Ba\textsuperscript{2+} (open) or 10 mM Ca\textsuperscript{2+} (filled).  
(Right) The peak current ratios (I\textsubscript{Ca}/I\textsubscript{Ba}) and relative slope conductance (G\textsubscript{MaxCa}/G\textsubscript{MaxBa}) for Ca-\alpha1T and Ca\textsubscript{v}3.1.
Student's t-test, ** \emph{p}$<$0.01, *** \emph{p}$<$0.001.
\textbf{(f)} Nickel inhibition sensitivity of Ca-\alpha1T and Ca\textsubscript{v}3.1. 
(Left) Representative current traces of Ca-\alpha1T and Ca\textsubscript{v}3.1 at the indicated Ni\textsuperscript{2+} concentrations. 
(Right) Dose-response curves indicating Ni\textsuperscript{2+}-dependent inhibition of Ca-\alpha1T ($\fullmoon$) and Ca\textsubscript{v}3.1 ($\square$) obtained by fitting the averaged data with the Hill equation.
Data are presented as means $\pm$ standard error of the mean (SEM).
