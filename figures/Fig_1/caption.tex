\label{fig:Fig}

{\bf Comparison of biophysical properties of DmCa\textsubscript{v}3 and rat Ca\textsubscript{v}3.1.}\\
{\bf(a)} (Left) Representative current traces through DmCa\textsubscript{v}3 and Ca\textsubscript{v}3.1 expressed in \emph{Xenopus} oocytes. 
In 10 mM Ba\textsuperscript{2+}, currents were elicited by depolarizing step pulses in 10 mV increments from \textminus70 mV to +40 mV from a holding potential of \textminus90 mV. 
(Right) I-V relationships of DmCa\textsubscript{v}3 and Ca\textsubscript{v}3.1. 
Peak currents for each oocyte were normalized to the maximum current recorded. 
Averaged percent amplitudes (mean $\pm$ s.e.m.) from oocytes expressing DmCa\textsubscript{v}3 (\raisebox{3pt}{\scalebox{2}{$\circ$}}) or Ca\textsubscript{v}3.1 ($\square$) are plotted against test potentials and fitted with Boltzmann equation. 
{\bf(b)} (Left) Steady-state inactivation was measured during voltage steps to \textminus20 mV after 10 sec prepulse to potentials between \textminus100 mV and \textminus40 mV. 
(Right) Voltage-dependent activation and steady-state inactivation curves of DmCa\textsubscript{v}3 channel (\raisebox{3pt}{\scalebox{2}{$\circ$}}, \raisebox{3pt}{\scalebox{2}{$\bullet$}}) and Ca\textsubscript{v}3.1 ($\square$, $\blacksquare$) were plotted and fitted with Boltzmann equation. 
{\bf(c)} The activation time constant ($\tau$\textsubscript{act}) and inactivation time constant ($\tau$\textsubscript{inact}) of DmCa\textsubscript{v}3 (\raisebox{3pt}{\scalebox{2}{$\circ$}}) and Ca\textsubscript{v}3.1 ($\square$) were obtained by fitting the current traces with double exponentials simultaneously. 
{\bf(d)} Voltage-dependent deactivation of DmCa\textsubscript{v}3 in HEK-293 cells. 
Tail currents were elicited by application of repeated step pulses to \textminus20 mV for 10 ms, followed by various re-polarizingjavascript:void(0) potentials from \textminus120 mV to \textminus50 mV. 
Deactivation time constants were obtained from fitting tail current traces with a single exponential and plotted against re-polarizing potentials (n=6). 
{\bf(e)} I\textsubscript{Ca}/I\textsubscript{Ba} ratios of DmCa\textsubscript{v}3 and Ca\textsubscript{v}3.1. 
(Left) Representative current traces through DmCa\textsubscript{v}3 and Ca\textsubscript{v}3.1 measured in 10 mM Ba\textsuperscript{2+} or 10 mM Ca\textsuperscript{2+} were elicited by \textminus10 mV step pulses from a holding potential of \textminus90 mV. 
Ba\textsuperscript{2+} currents are represented in black and Ca\textsuperscript{2+} currents are in grey. 
(Middle) I-V relationships of DmCa\textsubscript{v}3 (\raisebox{3pt}{\scalebox{2}{$\circ$}}, \raisebox{3pt}{\scalebox{2}{$\bullet$}}) and Ca\textsubscript{v}3.1 ($\square$, $\blacksquare$) in 10 mM Ba\textsuperscript{2+} (open) or 10 mM Ca\textsuperscript{2+} (filled) solution are shown.  
(Right) The ratios (I\textsubscript{Ca}/I\textsubscript{Ba}) of peak current amplitude through DmCa\textsubscript{v}3 and Ca\textsubscript{v}3.1 (n=5, 4) and their relative slope conductance (G\textsubscript{MaxCa}/G\textsubscript{MaxBa}) in 10 mM Ca\textsuperscript{2+} and 10 mM Ba\textsuperscript{2+} through DmCa\textsubscript{v}3 and Ca\textsubscript{v}3.1 (n=6, 4). Student's t-test, **p$<$0.01, ***p$<$0.001.
{\bf(f)} Nickel inhibition sensitivity of DmCa\textsubscript{v}3 and Ca\textsubscript{v}3.1. 
(Left) Representative current traces of DmCa\textsubscript{v}3 and Ca\textsubscript{v}3.1 after serial concentrations of nickel solutions. 
(Right) Dose-response curves of nickel inhibition on DmCa\textsubscript{v}3 (\raisebox{3pt}{\scalebox{2}{$\circ$}}) and Ca\textsubscript{v}3.1 ($\square$) were obtained from fitting the averaged data with the Hill equation.

  
  
  
  
  
  
  
  
  
  
  
  
  
  
  
  
  
  
  
  
  
  
  
  
  
  
  
  