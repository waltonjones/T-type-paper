\label{fig:3}
\textbf{Sleep is increased in DmCa\textsubscript{v}3 mutants}
\\
\textbf{A.} Western blot analysis of DmCa\textsubscript{v}3 protein levels in DmCa\textsubscript{v}3\textsuperscript{Gal4} and DmCa\textsubscript{v}3\textsuperscript{Rescue}.
DmCa\textsubscript{v}3 protein is undetectable in DmCa\textsubscript{v}3\textsuperscript{Gal4} while DmCa\textsubscript{v}3\textsuperscript{Rescue} shows a similar level as the w\textsuperscript{1118} control.
$\beta$-actin was used for internal control.
\textbf{B.} Sleep profiles of w\textsuperscript{1118} (black, n=89), DmCa\textsubscript{v}3\textsuperscript{Gal4} (red, n=92) and DmCa\textsubscript{v}3\textsuperscript{Rescue} (gray, n=61) over two days of 12h:12h light-dark cycle (LD) and two days of continuous dark condition (DD).
Sleep in 30 minute intervals was plotted.
Data are represented as mean $\pm$ sem.
White, black and gray bar denote light phase, dark phase and subjective light phase, respectively.
ZT, zeitgeber time.
CT, circadian time.
\textbf{C.} Averaged daily sleep amount in LD and DD.
\textbf{D.} Waking activity in LD and DD.
Waking activity is total activity counts divided by waking minutes.
\textbf{(E-G)} Sleep amount \textbf{E.}, the number of sleep bouts \textbf{F.} and average length of a sleep bout \textbf{G.} in light phase (L) and dark phase (D).
\textbf{(H-J)} Sleep amount \textbf{H.}, the number of sleep bouts \textbf{I.} and average length of a sleep bout \textbf{J.} in subjective light phase (SL) and subjective dark phase (SD).
In box-and-whisker plots, midline, upper and lower boundaries of each box represent median, 75\textsuperscript{th} and 25\textsuperscript{th} percentiles, respectively.
Upper and lower whiskers extend to the highest and lowest value within 1.5 $\times$ interquartile range of the upper and lower quartiles, respectively.
For statistical analysis, Kruskal-Wallis test followed by Dunn's multiple comparison test was performed.
*$p\le$0.01.
  
  