\label{fig:3}
\textbf{Sleep is increased in \emph{DmCa\textsubscript{v}3} mutants}
\\
\textbf{(a)} \emph{DmCa\textsubscript{v}3}, \emph{DmCa\textsubscript{v}3\textsuperscript{Gal4}}, and \emph{DmCa\textsubscript{v}3\textsuperscript{Rescue}} schematics. 
DmCa\textsubscript{v}3 coding exons are red.
Downward arrows denote the extent of the deleted region.
SA, splice acceptor.
pA, polyA sequence.  
\textbf{(b)} Western blot analysis of DmCa\textsubscript{v}3 protein levels of fly head lysates.
DmCa\textsubscript{v}3 is undetectable in \emph{DmCa\textsubscript{v}3\textsuperscript{Gal4}} lysates while \emph{DmCa\textsubscript{v}3\textsuperscript{Rescue}} lysates show levels similar to the \emph{w\textsuperscript{1118}} control.
$\beta$-actin was used as a loading control.
\textbf{(c)} Sleep profiles of \emph{w\textsuperscript{1118}} (black, n=89), \emph{DmCa\textsubscript{v}3\textsuperscript{Gal4}} (red, n=92) and \emph{DmCa\textsubscript{v}3\textsuperscript{Rescue}}  (grey, n=61) over two days of 12h:12h light-dark (LD) and two days of continuous dark (DD) conditions.
Sleep is plotted in 30 minute intervals.
Data are represented as mean $\pm$ s.e.m.
White, black, and grey bars denote light phase, dark phase, and subjective light phase, respectively.
ZT, zeitgeber time.
CT, circadian time.
\textbf{(d)} Total daily sleep under LD and DD conditions.
\textbf{(e)} Waking activity under LD and DD conditions measured as total activity counts divided by waking minutes.
\textbf{(f)} The number of sleep bouts under LD and DD conditions.
\textbf{(g)} Average sleep bout length under LD and DD conditions.
Boxplot whiskers extend to the highest and lowest values that fall within 1.5 $\times$ IQR of the upper and lower quartiles.
All indications of statistical significance were determined using Welch's ANOVA followed by the Games-Howell post hoc test.
* \emph{p}$<$0.05, ** \emph{p}$<$0.01, *** \emph{p}$<$0.001.

  
  