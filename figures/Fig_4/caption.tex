\label{fig:4}
\textbf{ \emph{DmCa\textsubscript{v}3\textsuperscript{Gal4}} flies show rhythmic circadian locomotion with normal transcriptional oscillation of \emph{period} and normal homeostatic regulation.}
\\
\textbf{(a)} Average activity profiles from day 2 of the 12h:12h light-dark cycles (LD, left), day 2 of continuous darkness (DD, middle), and from throughout the experiment (2 LD + 7 DD, right).
In the left and middle panels, data are presented as means $\pm$ standard error of the mean (SEM).
In the right panel, white, black, and grey bars indicate light phase, dark phase, and subjective light phase, respectively.
The dotted line indicates the beginning of constant darkness. 
The number of flies measured, their rhythmic period, their power of rhythmicity (P-S), and the percentage of rhythmic flies (Rhythmicity) are indicated.
a.u., arbitrary unit.
The Mann-Whitney U test was used to determine the significance of the period changes (*\emph{p}$<$0.05), while Welch's t-test was used for rhythmic power (***\emph{p}$<$0.001). 
\textbf{(b)} Transcriptional oscillation of the \emph{period} gene in \emph{DmCa\textsubscript{v}3\textsuperscript{Gal4}} under DD conditions. Black and red lines denote \emph{w\textsuperscript{1118}} and \emph{DmCa\textsubscript{v}3\textsuperscript{Gal4}}, respectively.
\emph{rp49} was used for normalization.
a.u., arbitrary unit.
\textbf{(c)} Percentage of lost sleep recovered (\% $\Delta$Sleep) over a 12 hr period after 24 hours of mechanically-induced sleep deprivation. \emph{w\textsuperscript{1118}} (n=35) and \emph{DmCa\textsubscript{v}3\textsuperscript{Gal4}} (n=33). 
Statistical significance was determined using the Student's t-test. 
ns, not significant.
Data are presented as means $\pm$ standard error of the mean (SEM).
