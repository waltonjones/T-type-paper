\label{fig:S3}
\textbf{Loss of DmCa\textsubscript{v}3 increases sleep in constant darkness.}
\\
\textbf{(a)} The DmCa\textsubscript{v}3 locus indicating the regions deleted in the \emph{$\Delta$3}, \emph{$\Delta$115} and \emph{$\Delta$135} mutants.
The insertion point of the EP line (G1047) is indicated.
DmCa\textsubscript{v}3 exons are represented in black.
The bracketed areas represent the deleted regions for each mutant and the deletion sizes are indicated.
\textbf{(b)} Western blot analysis of DmCa\textsubscript{v}3 protein levels in \emph{w\textsuperscript{1118}}, \emph{$\Delta$3}, \emph{$\Delta$115} and \emph{$\Delta$135}.
DmCa\textsubscript{v}3 protein is undetectable in lysates from all 3 deletion mutants.
$\beta$-actin was used as a loading control.
\textbf{(c)} Sleep profiles of \emph{w\textsuperscript{1118}} (n=31), \emph{$\Delta$3} (n=31), \emph{$\Delta$115} (n=32) and transhetero mutant \emph{$\Delta$3/$\Delta$115} (n=32) over two days of 12h:12h light-dark cycles (LD) and two days of continuous darkness (DD).
Sleep is plotted in 30 minute intervals.
White, black, and grey bars denote light phase, dark phase and subjective light phase, respectively.
ZT, zeitgeber time.
CT, circadian time.
\textbf{(d)} Average total sleep of \emph{DmCa\textsubscript{v}3} mutants in the light (L) and dark phases (D) over two days of LD (left) and in the subjective light (SL) and subjective dark (SD) phases over two days of DD (right).
\emph{w\textsuperscript{1118}} (n=31), \emph{$\Delta$3} (n=31), \emph{$\Delta$115} (n=32), \emph{$\Delta$135} (n=29), \emph{$\Delta$3/$\Delta$115} (n=32), \emph{$\Delta$3/$\Delta$135} (n=32), and \emph{$\Delta$115/$\Delta$135} (n=30).
Data are presented as means $\pm$ standard error of the mean (SEM).
Significance was determined with Welch's t-test.
*\emph{p}$<$0.05, **\emph{p}$<$0.01, ***\emph{p}$<$0.001.
