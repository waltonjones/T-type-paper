\label{fig:S2}
\textbf{Deletion mutants of DmCa\textsubscript{v}3 increases sleep in constant darkness.}
\\
\textbf{A.} Schematic representation of DmCa\textsubscript{v}3 gene locus and deleted region in mutants, $\Delta$3, $\Delta$115 and $\Delta$135.
G1047 line was used for P-element imprecise excision.
Exons are denoted by black squares.
Bracketed area represents deleted regions and deletion size are indicated within the brackets.
\textbf{B.} Western blot analysis of DmCa\textsubscript{v}3 protein levels in w\textsuperscript{1118}, $\Delta$3, $\Delta$115 and $\Delta$135.
DmCa\textsubscript{v}3 protein is not detectable in all deletion mutants.
$\beta$-actin was used for internal control.
\textbf{C.} Sleep profiles of w\textsuperscript{1118} (n=31), $\Delta$3 (n=31), $\Delta$115 (n=32) and transhetero mutant $\Delta$3/$\Delta$115 (n=32) over two days of 12h:12h light-dark cycle (LD) and two days of continuous dark condition (DD).
Sleep in 30 minute intervals was plotted.
White, black and gray bar denote light phase, dark phase and subjective light phase, respectively.
ZT, zeitgeber time.
CT, circadian time.
\textbf{D.} Averaged sleep amount in light (L)/dark phase (D) in two days of LD (left) and subjective light (SL)/subjective dark (SD) phase in two days of DD (right) in DmCa\textsubscript{v}3 mutants.
w\textsuperscript{1118} (n=31), $\Delta$3 (n=31), $\Delta$115 (n=32), $\Delta$135 (n=29), $\Delta$3/$\Delta$115 (n=32), $\Delta$3/$\Delta$135 (n=32), and $\Delta$115/$\Delta$135 (n=30).
Data are represented as mean $\pm$ sem.
For statistical analysis, Mann-Whitney U test was performed.
*$p\le$0.05.
  