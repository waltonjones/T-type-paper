\label{fig:S2}
\textbf{Loss of DmCa\textsubscript{v}3 increases sleep in constant darkness.}
\\
\textbf{(a)} The DmCa\textsubscript{v}3 locus indicating the regions deleted in the \emph{$\Delta$3}, \emph{$\Delta$115} and \emph{$\Delta$135} mutants.
The insertion point of the EP line (G1047) from the \href{http://genexel.kaist.ac.kr}{Genexel collection} used for imprecise P-element excision is indicated.
DmCa\textsubscript{v}3 exons are represented in black.
Bracketed area represents deleted regions and deletion size are indicated within the brackets.
\textbf{(b)} Western blot analysis of DmCa\textsubscript{v}3 protein levels in \emph{w\textsuperscript{1118}}, \emph{$\Delta$3}, \emph{$\Delta$115} and \emph{$\Delta$135}.
DmCa\textsubscript{v}3 protein is not detectable in all deletion mutants.
$\beta$-actin was used for internal control.
\textbf{(c)} Sleep profiles of \emph{w\textsuperscript{1118}} (n=31), \emph{$\Delta$3} (n=31), \emph{$\Delta$115} (n=32) and transhetero mutant \emph{$\Delta$3/$\Delta$115} (n=32) over two days of 12h:12h light-dark cycle (LD) and two days of continuous dark condition (DD).
Sleep in 30 minute intervals was plotted.
White, black and grey bar denote light phase, dark phase and subjective light phase, respectively.
ZT, zeitgeber time.
CT, circadian time.
\textbf{(d)} Averaged sleep amount in light (L)/dark phase (D) in two days of LD (left) and subjective light (SL)/subjective dark (SD) phase in two days of DD (right) in \emph{DmCa\textsubscript{v}3} mutants.
\emph{w\textsuperscript{1118}} (n=31), \emph{$\Delta$3} (n=31), \emph{$\Delta$115} (n=32), \emph{$\Delta$135} (n=29), \emph{$\Delta$3/$\Delta$115} (n=32), \emph{$\Delta$3/$\Delta$135} (n=32), and \emph{$\Delta$115/$\Delta$135} (n=30).
Data are presenteds as means $\pm$ s.e.m..
For statistical analysis, Welch's t-test was performed.
*p$<$0.05, **p$<$0.01, ***p$<$0.001.
  
  
  