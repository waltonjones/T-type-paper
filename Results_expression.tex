\subsection*{DmCa\textsubscript{v}3 is broadly expressed in the adult brain}

After several failed attempts to generate an antibody that works well for immunohistochemistry, we decided to tag the endogenous DmCa\textsubscript{v}3 with GFP and then visualize its expression pattern in the adult brain.
First, we generated a founder line, (\emph{DmCa\textsubscript{v}3\textsuperscript{Founder, \emph{w+}}}}), using end-out homologous recombination to facilitate the versatile generation of a variety of different alleles\cite{Huang:2009ei} (Fig. \ref{fig:2}a).
In \emph{DmCa\textsubscript{v}3\textsuperscript{Founder, \emph{w+}}}} flies, an \emph{attP} landing site for $\phi$C31-mediated DNA integration and a floxed \emph{white\textsuperscript{+}} marker replace \sim2 kb of genomic DNA surrounding the first coding exon of DmCa\textsubscript{v}3. 
Next, we removed the \emph{white\textsuperscript{+}} marker from the \emph{DmCa\textsubscript{v}3\textsuperscript{Founder, \emph{w+}}} line by Cre-mediated recombination to generate \emph{DmCa\textsubscript{v}3\textsuperscript{Founder, \emph{w-}}}.
We then used the $\phi$C31 integrase to insert into the \emph{attP} landing site of \emph{DmCa\textsubscript{v}3\textsuperscript{Founder, \emph{w-}}} an \emph{attB}vector (\emph{pGE-attB\textsuperscript{GMR}}) containing the deleted genomic region plus an additional GFP coding sequence and linker sequence in-frame before the start codon of DmCa\textsubscript{v}3. 
This produced the \emph{GFP::DmCa\textsubscript{v}3} line, which expresses an N-terminally GFP-tagged DmCa\textsubscript{v}3 under the control of its own endogenous promoter.

In these \emph{GFP::DmCa\textsubscript{v}3} flies, GFP fluorescence appears broadly across the brain (Fig. \ref{fig:2}b).
\emph{GFP::DmCa\textsubscript{v}3} is expressed in well-structured neuropils like the antennal lobes, the mushroom bodies, the central complex (Fig. \ref{fig:2}c-h), the optic lobes, as well as in some of the less-structured neuropils.
The central complex---comprising the fan-shaped body, ellipsoid body, noduli, and protocerebral bridge---shows the strongest expression with the ventral fan-shaped body and ventral noduli particularly prominent (Fig. \ref{fig:2}e and g).
In mushroom body neurons, there is far more \emph{GFP::DmCa\textsubscript{v}3} in the dendrite-rich calyx of the dorso-posterior brain (Fig. \ref{fig:2}h) than the axon-rich lobes of the anterior brain (Fig. \ref{fig:2}d).
\emph{GFP::DmCa\textsubscript{v}3} is also limited to the posterior mushroom body peduncles, which are the fiber tracks that join the posterior calyces with the anterior mushroom body lobes (Fig. \ref{fig:2}f).
These results suggest strict regulation of the subcellular localization of DmCa\textsubscript{v}3 channels in the brain.

We next visualized the projections of DmCa\textsubscript{v}3-expressing neurons using another knock-in allele, \emph{DmCa\textsubscript{v}3\textsuperscript{Gal4}}.
In \emph{DmCa\textsubscript{v}3\textsuperscript{Gal4}}, the first coding exon and flanking introns of DmCa\textsubscript{v}3 are replaced by the Gal4 coding sequence.
This puts GAL4 expression under the control of the endogenous DmCa\textsubscript{v}3 promoter (Fig. \ref{fig:3}a).
Consistent with our results using \emph{GFP::DmCa\textsubscript{v}3}, \emph{DmCa\textsubscript{v}3\textsuperscript{Gal4}} drives the expression of a membrane-tethered mCherry (\emph{UAS-mCD8-ChRFP}) broadly across the brain (Fig. \ref{fig:S1}a).
The \emph{DmCa\textsubscript{v}3\textsuperscript{Gal4}\textgreater{}mCherry} and \emph{GFP::DmCa\textsubscript{v}3} signals are strongly co-localized, including in the central complex and mushroom bodies (\ref{fig:S1}b and c}).
This suggests both reagents reflect proper expression from the same endogenous DmCa\textsubscript{v}3 promoter.

  
  
  
  
  