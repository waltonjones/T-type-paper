\section*{Introduction}

T-type Ca\textsuperscript{2+} channels are a subfamily of voltage-dependent Ca\textsuperscript{2+} channels (VDCCs) that produce low-voltage-activated (LVA) Ca\textsuperscript{2+} currents implicated in NREM sleep in mammals\cite{Lee:2004ey}.
Three different genes encode the pore-forming alpha subunits of mammalian T-type channels, Ca\textsubscript{v}3.1, 3.2, and 3.3. Of these, Ca\textsubscript{v}3.1 and 3.3 are highly expressed in the thalamus, where the oscillations required for NREM sleep are generated\cite{PerezReyes:1998gn}.\textcolor{red}{More citation needed for NREM oscillation in thalamus?}
Mice lacking Ca\textsubscript{v}3.1 show reduced delta-wave activity and reduced sleep stability, suggesting that mammalian T-type currents have a sleep-promoting or stabilizing function\cite{Lee:2004ey}.

Unlike mammals, \emph{Drosophila melanogaster} have only one T-type Ca\textsuperscript{2+} channel, DmCa\textsubscript{v}3, which is also known as Dm$\alpha$G and $\alpha$1T.
A recent study found that motor neurons in flies lacking DmCa\textsubscript{v}3 show reduced LVA but also high-voltage-activated (HVA) Ca\textsuperscript{2+} currents, suggesting that although DmCa\textsubscript{v}3 seems to be a genuine T-type channel, it may have interesting biophysical properties\cite{Ryglewski:2012jk}.
We therefore cloned a single isoform of DmCa\textsubscript{v}3,  expressed it in \emph{Xenopus} oocytes, and compared its biophysical properties with those of the rat T-type channel DmCa\textsubscript{v}3.1.
We also generated several DmCa\textsubscript{v}3 mutant alleles and identified a defect in their sleep/wake cycles. Contrary to results in mammals, the fly T-type Ca\textsuperscript{2+} channel destabilizes sleep.
We anticipate that our findings will help clarify species-dependent differences in the \emph{in vivo} functions of T-type Ca\textsuperscript{2+} channel.
    
    
  
  
  
  
  