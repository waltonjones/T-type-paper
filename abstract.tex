\section*{Abstract}
Mammalian T-type Ca^ {2+} channels, which are encoded by three genes (Ca_{v}3.1, 3.2, 3.3), mediate the induction of sleep oscillations that promote sleep stability.
Invertebrates have only one T-type Ca^ {2+} channel gene and its physiological functions are less well-studied.
We cloned DmCa_{v}3, the only fly Ca_{v}3 channel gene, and found that it shows broad expression across the brains of \emph{Drosophila melanogaster} adults.
Voltage-clamp analysis revealed that the biophysical and pharmacological properties of DmCa_{v}3 are more similar to Ca_{v}3.2 and Ca_{v}3.3 than Ca_{v}3.1.
Flies lacking DmCa_{v}3 show an abnormal increase in sleep duration that is most pronounced during subjective day under continuous dark conditions despite normal oscillation of the circadian clock.
Our study suggests that invertebrate T-type Ca^{2+} channels promote wakefulness rather than stabilizing sleep like their vertebrate counterparts.
    