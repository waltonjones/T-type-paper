\section*{Abstract}

Mammalian T-type Ca\textsuperscript{2+} channels are encoded by three separate genes (Ca\textsubscript{v}3.1, 3.2, 3.3).
Progress in identifying the physiological functions of the T-type channels has been hindered by many factors, including possible compensation between the products of these three genes and a lack of specific pharmacological inhibitors.
Still, sleep has emerged as a major physiological process regulated by T-type currents.
Invertebrates have only one T-type Ca\textsuperscript{2+} channel gene and its physiological functions are less well-studied.
We cloned DmCa\textsubscript{v}3, the only fly Ca\textsubscript{v}3 channel gene, and found that it shows broad expression across the brains of \emph{Drosophila melanogaster} adults.
Voltage-clamp analysis revealed that the biophysical and pharmacological properties of DmCa\textsubscript{v}3 are more similar to Ca\textsubscript{v}3.2 and Ca\textsubscript{v}3.3 than Ca\textsubscript{v}3.1.
Flies lacking DmCa\textsubscript{v}3 show an abnormal increase in sleep duration that is most pronounced during subjective day under continuous dark conditions despite normal oscillations of the circadian clock.
Our study suggests that invertebrate T-type Ca\textsuperscript{2+} channels promote wakefulness rather than stabilizing sleep like their vertebrate counterparts.
  
  
  