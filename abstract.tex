\section*{Abstract}

Mammalian T-type Ca\textsuperscript{2+} channels are encoded by three separate genes (Ca\textsubscript{v}3.1, 3.2, 3.3).
In mammals, T-type channels are reported to be sleep stabilizers that are important in the generation of the delta rhythms of deep sleep, but controversy remains.
Progress in identifying the precise physiological functions of the T-type channels has been hindered by many factors, including possible compensation between the products of these three genes and a lack of specific pharmacological inhibitors.
Invertebrates have only one T-type channel gene and its physiological functions are less well-studied.
We cloned DmCa\textsubscript{v}3, the only Ca\textsubscript{v}3 channel gene in the \emph{Drosophila melanogaster} genome, expressed it in \emph{Xenopus} oocytes, and verified that it is capable of passing typical T-type currents.
Voltage-clamp analysis revealed that the biophysical properties of DmCa\textsubscript{v}3 are more similar to Ca\textsubscript{v}3.2 and Ca\textsubscript{v}3.3 than Ca\textsubscript{v}3.1.
We found that DmCa\textsubscript{v}3 is broadly expressed across the adult fly brain in a pattern vaguely reminiscent of mammalian T-type channels.
In addition, flies lacking DmCa\textsubscript{v}3 show an abnormal increase in sleep duration that is most pronounced during subjective day under continuous dark conditions despite normal oscillations of the circadian clock.
Thus, our study suggests invertebrate T-type Ca\textsuperscript{2+} channels promote wakefulness rather than stabilizing sleep.
