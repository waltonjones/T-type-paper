\subsection*{Electrophysiological recordings and data analysis}

Barium (or Ca\textsupercript{2+}) currents through T-type channels expressed in oocytes were measured at room temperature 4 to 5 days after cRNA injection using a two-electrode voltage-clamp amplifier (OC-725C, Warner Instruments, Hamden, CT, USA).
Microelectrodes were pulled from the capillaries (Warner Instruments, Hamden, CT, USA) using a pipette puller and filled with 3 M KCl and the electrode resistance was 0.5-1.1 M$\Omega$.
The 10 mM Ba\textsuperscript{2+} (or Ca\textsupercript{2+}) recording solution contained (in mM): 10 Ba(OH)2 (or Ca(OH)2), 90 NaOH, 1 KOH, 5 HEPES (pH 7.4 adjusted with methanesulfonic acid).
To get rid of contamination of Ca\textsupercript{2+}-activated chloride currents, we injected 50 nl of 50 mM BAPTA into oocytes 30-60 min before recordings, especially for recording Ca\textsupercript{2+} currents from oocytes.
The currents were usually sampled at 5 kHz and low pass filtered at 1 kHz using the pClamp system (Digidata 1320A and pClamp 8; Axon instruments, Foster City, CA, USA). 
For recordings of tail currents of DmCav3, we employed whole cell patch clamp recordings in the HEK-293 cells transiently transfected with DmCav3. Recordings were obtained at room temperature using an Axopatch 200A amplifier, which was connected to a computer through a Digidata 1300 A/D converter, and controlled using pCLAMP 9.2 software. Currents were recorded in a 10 mM Ba2+ solution (in mM): 140 TEACl, 2.5 CsCl, 10 BaCl2, 1 MgCl2, 10 HEPES, 10 glucose, pH = 7.3 adjusted with TEAOH. The pipette solution contained the following (in mM): 130 CsCl, 10 HEPES, 10 EGTA, 5 MgATP, 1 NaGTP, pH = 7.4 adjusted with CsOH. Recording pipettes were prepared from TW-150-3 capillary (World Precision Instruments, Inc., Sarasota, FL). The pipette resistance was 2.0 ~ 3.0 M. Access resistance was compensated 70-80% using the compensation circuit and series resistance prediction. Data for tail currents were filtered at 10 kHz and digitized at 20 kHz. 
Peak currents and exponential fits to currents were analyzed using Clampfit software (Axon instruments, Foster City, CA, USA) and graphical presentation of the data was prepared using Prism software (GraphPad, San Diego, CA, USA). 
Data are presented as mean $\pm$ s.e.m.
The levels of significant difference(s) between groups were tested using Student's unpaired t-test: $p\le$0.05, $p\le$0.01, and $p\le$0.001 as the level of significance.
Activation and inactivation time constants of T-type channel currents elicited by step pulses were estimated by fitting individual current traces with a double exponential function: A1(1-exp(-t/$\tau$1)) + A2(1-exp(-t/$\tau$2)) where A1 and A2 are the coefficients for the activation and inactivation exponentials, t is time, and $\tau$1 and $\tau$2 are the activation and inactivation time constants, respectively.
The smooth curves for channel activation and steady-state inactivation were from fitting data with a Boltzmann equation: , where V50 is the potential for half-maximal activation and Sact is the slope conductance.
Dose-response curves for Ni\textsuperscript{2+} inhibition of T-type channel currents were derived by fitting the data using a Hill equation: B = 1/{1 + IC\textsubscript{50}/[Cd2+]n}, where B is the normalized block, IC\textsubscript{50} is the concentration of Ni\textsuperscript{2+} giving half maximal blockade, and n is the Hill coefficient.
    
  