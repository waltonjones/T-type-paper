\subsection*{Electrophysiology}

Ba\textsuperscript{2+} (or Ca\textsuperscript{2+}) currents through T-type channels expressed in oocytes were measured at room temperature 4--5 days after cRNA injection using a two-electrode voltage-clamp amplifier (OC-725C, Warner Instruments, Hamden, CT, USA).
Microelectrodes were pulled from capillaries (G100TF-4, Warner Instruments, Hamden, CT, USA) using a pipette puller and filled with 3 M KCl.
All electrodes used measured 0.5--1.1 M$\Omega$ of resistance.
The 10 mM Ba\textsuperscript{2+} (or Ca\textsuperscript{2+}) recording solution contained: 10 mM Ba(OH)\textsubscript{2} (or Ca(OH)\textsubscript{2}), 90 mM NaOH, 1 mM KOH, 5 mM HEPES (pH 7.4, adjusted with methanesulfonic acid).
To remove any contamination from Ca\textsuperscript{2+}-activated chloride currents, we injected the oocytes with 50 nL of 50 mM BAPTA (1,2-bis$[$o-aminophenoxy$]$ ethane -N,N,N\textquoteright,N\textquoteright-tetraacetic acid) 30--60 min prior to recording.
This was especially important while recording Ca\textsuperscript{2+} currents.
Currents were sampled at 5 kHz and low pass filtered at 1 kHz using the pClamp system (Digidata 1320A and pClamp 8; Axon instruments, Foster City, CA, USA) unless otherwise noted. 

We used whole cell patch clamp recordings from HEK-293 cells transiently transfected with DmCa\textsubscript{v}3 to measure tail currents. 
These recordings were obtained at room temperature using an Axopatch 200A amplifier connected to a computer through a Digidata 1300 A/D converter and controlled with the pCLAMP 9.2 software.
Tail currents were recorded in a 10 mM Ba\textsuperscript{2+} solution containing the following: 140 mM TEACl, 2.5 mM CsCl, 10 mM BaCl\textsubscript{2}, 1 mM MgCl\textsubscript{2}, 10 mM glucose, and 10 mM HEPES (pH 7.3, adjusted with TEAOH).
The pipette solution contained the following: 130 mM CsCl, 10 mM EGTA, 5 mM MgATP, 1 mM NaGTP, and 10 mM HEPES (pH 7.4, adjusted with CsOH). 
Recording pipettes were prepared from TW-150-3 capillaries (World Precision Instruments, Inc., Sarasota, FL).
The pipette resistance was 2.0$\sim$3.0 M$\Omega$.
Access resistance was compensated by 70--80\% using the compensation circuit and series resistance prediction.
Tail current data were filtered at 10 kHz and digitized at 20 kHz.
Peak currents and exponential fits were analyzed using the Clampfit software package (Axon instruments, Foster City, CA, USA).
The activation and inactivation time constants for the T-type currents elicited by step pulse protocols were estimated by fitting individual current traces with double exponential functions: $A_{1}(1-exp(-t/\tau_{1})) + A_{2}(1-exp(-t/\tau_{2}))$ where $A_{1}$ and $A_{2}$ are the coefficients for the activation and inactivation exponentials, $t$ is time, and $\tau_{1}$ and $\tau_{2}$ are the activation and inactivation time constants, respectively.
The smooth curves for channel activation and steady-state inactivation were obtained by fitting the data with a Boltzmann equation: $1/\{1+exp[(V_{50}-V)/S_{act}]\}$, where $V_{50}$ is the potential for half-maximal activation and $S_{act}$ is the slope conductance.
Dose-response curves for Ni\textsuperscript{2+} inhibition of T-type channel currents were derived by fitting the data using a Hill equation: $B = 1/(1 + \textrm{IC}_{50}/[\textrm{Ni}^{2+}]^n)$, where $B$ is the normalized block, $\textrm{IC}_{50}$ is the concentration of $\textrm{Ni}^{2+}$ giving half maximal blockade, and $n$ is the Hill coefficient.
