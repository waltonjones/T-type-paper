\subsection*{Sleep and locomotor behavior analysis}

Fly sleep and locomotor behavior was measured with the \emph{Drosophila} Activity Monitoring system (Trikinetics).
For sleep analysis, 3-4 day-old female flies were placed individually into 65 mm X 5 mm glass tubes with one end filled with 2\% agar / 5\% sucrose food and the other end plugged with cotton.
We defined periods of activity as periods with a beam break frequency higher than 1 per minute and periods of sleep as periods during which no beam break occurred for at least 5 consecutive minutes\cite{Shaw:2000ui}.
After one day of habituation in an incubator (25$\,^{\circ}\mathrm{C}$, 60\% humidity), we used the ``Counting Macro'' software\cite{pfeiffenberger:2010ab} to measure sleep over the course of 4 days---2 days of 12 hr:12 hr light-dark conditions and 2 days of continuous darkness.
For experiments using the GeneSwitch technique, flies were maintained on normal food containing 500 $\mu$M RU486 (M8046, Sigma-Aldrich) dissolved in ethanol (1\%) for two days prior to the experiment.
Control flies were maintained on normal food containing only ethanol (1\%).
During the GeneSwitch experiments, flies were placed in 2\% agar / 5\% sucrose food with or without 500 $\mu$M RU486.
For sleep deprivation experiment, activity monitors with 3-5 day-old flies were placed in the apparatus designed to rotate and give a mechanical stimuli about twice a minute.
After 2 days of habituation  
For the circadian locomotor analyses, we measured the activity of 1-3 day-old male flies in 30 minute bins and analyzed the data using ClockLab (Actimetrics) and the Counting Macro\cite{pfeiffenberger:2010aa}. Significance level for the $\chi$\textsuperscript{2} periodogram was set to $\alpha$ = 0.05.
Flies with a power of significance (P-S) $\geq$10 were considered rhythmic.

  