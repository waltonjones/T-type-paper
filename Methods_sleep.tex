\subsection*{Sleep and locomotor behavior analysis}

Sleep and locomotor behavior of flies were measured using the \emph{Drosophila} Activity Monitoring system (Trikinetics).
For sleep analysis, 3-4 day-old female flies were placed individually into 65 mm X 5mm glass tubes with one end filled with 2\% agar/5\% sucrose food and the other end plugged with cotton.
Periods of activity were defined as periods with a beam break frequency higher than 1 per minute, and periods of sleep were defined as periods with a beam break frequency lower than 1 per 5 minutes\cite{Shaw:2000ui}.
After one day of habituation in an incubator (25$\,^{\circ}\mathrm{C}$, 60\% humidity), sleep during two days of 12hr:12hr light-dark cycle and following two days of dark period was analysed using Counting Macro \cite{pfeiffenberger:2010ab}.
For circadian locomotor analysis, 1-3 day-old male flies were used. Activity was measured every 30 min bin and analysed using ClockLab analysis software (Actimetrics) and Counting Macro\cite{pfeiffenberger:2010aa}. Significance level of the $\chi$\textsuperscript{2} periodogram was set to $\alpha$ = 0.05.
Flies with a power of significance (P-S) $\geq$10 were considered as rhythmic.
